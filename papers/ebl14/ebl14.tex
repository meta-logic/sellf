\documentclass[a4paper,10pt]{article}
\usepackage[utf8]{inputenc}
\usepackage{mycommands}
\usepackage{myproof}
\usepackage{color}
\usepackage{stmaryrd}
\usepackage{latexsym}
\usepackage{url}
\usepackage{amsmath}
\usepackage{amssymb}
\usepackage{times}
% \usepackage{../mycommands}
\usepackage{multirow}
\usepackage{xspace}
\usepackage{alltt}
\usepackage{booktabs}
%\usepackage{lineno}
\usepackage{algorithm}
\usepackage{algorithmic}
\RequirePackage{txfonts}
\newcommand\ctx[2]{\ensuremath{\textrm{ctx}_R(#1,#2)}}

\usepackage{xspace}
\RequirePackage{txfonts}
\definecolor{MyDarkBlue}{RGB}{0,78,126}
\definecolor{MyRed}{RGB}{255,48,48}
\newcommand\thSeq{\ensuremath{\Tscr}}
\newcommand\groundSet{\ensuremath{\mathcal{B}}}

% \newcommand\leftarrow\supset
\newcommand\lra{\longrightarrow}

% Constraint predicates
\newcommand{\elin}[2]{\ensuremath{{\tsl{unitctx}(\ensuremath{#1}, \ensuremath{#2})}}}
\newcommand{\emp}[1]{\ensuremath{{\tsl{emp}(\ensuremath{#1})}}}
\newcommand{\eqctx}[2]{\ensuremath{{\tsl{eqctx}(\ensuremath{#1},\ensuremath{#2})}}}
\newcommand{\union}[3]{\ensuremath{{\tsl{union}(\ensuremath{#1},\ensuremath{#2},\ensuremath{ #3})}}}
\newcommand{\In}[2]{\ensuremath{\tsl{in}(\ensuremath{#1},\ensuremath{#2})}}
\newcommand{\Equ}[2]{\ensuremath{\tsl{Equ}(\ensuremath{#1},\ensuremath{#2})}}
\newcommand{\unions}[2]{\ensuremath{\tsl{Unions}(\ensuremath{#1},\ensuremath{#2})}}
\newcommand{\equal}[2]{\ensuremath{\tsl{Equal}(\ensuremath{#1},\ensuremath{#2})}}
\newcommand{\equalCtx}[2]{\ensuremath{\tsl{EqualCtx}(\ensuremath{#1},\ensuremath{#2})}}
% ProveIf predicates
\newcommand{\inSequent}[2]{\ensuremath{\tsl{inSequent}(\ensuremath{#1},\ensuremath{#2})}}
\newcommand{\inDer}[2]{\ensuremath{\tsl{inDer}(\ensuremath{#1},\ensuremath{#2})}}
\newcommand{\provIf}[2]{\ensuremath{\tsl{proveIf}(\ensuremath{#1},\ensuremath{#2})}}
\newcommand{\notProvIf}[2]{\ensuremath{\tsl{notProveIf}(\ensuremath{#1},\ensuremath{#2})}}
\newcommand{\bounded}[1]{\ensuremath{{\tsl{bounded}(\ensuremath{#1})}}}

\newtheorem{theorem}{Theorem} 
\newtheorem{definition}[theorem]{Definition}
\newtheorem{lemma}[theorem]{Lemma}
\newtheorem{proposition}[theorem]{Proposition}
\newtheorem{corollary}[theorem]{Corollary}
\newenvironment{remark}{\noindent \textbf{Remark}\quad}{}
\renewcommand{\red}[1]{\textcolor{red}{#1} }
\newenvironment{Paragraph}[1]{\paragraph{#1}}


% Names of the logic programs
\newcommand\LPder{\ensuremath{\mathbb{P}_1}}
\newcommand\LPprov{\ensuremath{\mathbb{P}_2}}
\newcommand\dnot{\ensuremath{\mathit{not}}\xspace}
%opening
\title{Quati: From Linear Logic Specifications to Inference Rules (Extended Abstract)}
\author{Vivek Nigam \& Leonardo Lima \\
Universidade Federal da Paraíba, Brazil\\
\texttt{vivek.nigam@gmail.com} \& \texttt{leonardo.alfs@gmail.com} 
\and Giselle Reis\\
Technische Universit\"at Wien, Vienna, Austria\\
\texttt{giselle@logic.at}}

\begin{document}

\maketitle

\begin{abstract}
In our previous work, we have shown that a great number of sequent calculus proof systems, such 
as a multiconclusion system for intuitionisitic logic, can 
be specified as theories in subexponential linear logic, a refinement of linear logic. 
These theories are natural, allowing, for example, the development of techniques for checking
automatically non-trivial properties of the encoded systems, such as whether the encoded proof system 
admits cut-elimination or which rule permutations are always allowed. 
However, it takes some effort to check whether a linear logic theory corresponds
indeed to a given proof system. This paper fills this gap by developing a technique that automatically 
transforms a linear logic specification into the sequent calculus inference rules it specifies thus allowing
the user to check for errors in the specification. We applied this technique to our previous work for 
checking and enumerating rule permutations from linear logic specifications, obtaining figures
as they would normally appear in a proof theory text book. We are currently implementing our technique 
in a tool called Quati to be launched soon.
\end{abstract}

\section{Introduction}
Quati is a tool to be launched soon that takes a specification in subexponential
linear logic~\cite{nigam09ppdp}, henceforth referred to as \sell, and infers
automatically rule permutations that are always allowed. The theory and
technique that are the basis of this tool is explained in our previous
work~\cite{nigam13iclp}. This paper describes one of Quati's features that was
not explained before, namely, how to construct sequent calculus rules as they
would appear in a standard Proof-Theory book~\cite{troelstra96bpt} from a
specification in \sell.

Subexponential linear logic is a refinement of linear logic which allows any
number of exponential-like connectives (i.e., $!$ and $?$), called
subexponentials, written $\nbang{\ell}$ and $\nquest{\ell}$ with a label $\ell$.
These labels are organized in a pre-order, $\prec$, which specifies the
provability relation among subexponentials. Moreover, some subexponentials allow
weakening and contraction of the formulas to which they are applied, these are
classified as unbounded. The remaining subexponentials do not allow weakening
and contraction of their formulas, these are classified as bounded. 
In our recent work~\cite{nigam.jlc}, we demonstrated that a number of proof
systems with rather complicated structural rules can be specified as \sell\
theories, e.g., the multiconclusion system \mLJ\ for intuitionisitic logic and a
proof system for the modal logic S4.
In fact our encoding is quite strong, having an adequacy on the level of 
derivations~\cite{nigam10jar}. This means that there is a one-to-one correspondence
between the (focused) derivations in \sell\ obtained from the theory and the
derivations in the encoded proof system.

This notion of adequacy is formalized by using a more advanced proof theory
concept called focusing, which we refrain to explain into details here for
readability purposes. Further details can be found in \cite{nigam.jlc}. In a
nutshell focusing is a complete discipline for proof search, introduced first in
the context of linear logic programming by Andreoli~\cite{andreoli92jlc}, which
reduces the proof search non-determinism. Focused proofs can be seen as a
normal-form for proof search. 

For our running example, we will use the implication right rule of the
multi-conclusion sequent calculus system for intuitionistic logic called
\mLJ~\cite{maehara54nmj}. We have shown in \cite{nigam.jlc} that the linear
logic formula $F$ to the left corresponds exactly to the rule depicted on the
right. In this setting, we say that intuitionistic logic is the object-level
logic, and \sell\ is the meta-level logic.
\[
F = \exists A. \exists B. [\rght{A\iimp B}^\bot \tensor \nbang{l}(\nquest{l} \lft{A}
\lpar \nquest{r} \rght{B})] \qquad \qquad 
 \infer[{\iimp_R}]{\Gamma \lra A \iimp B, \Delta}{\Gamma, A
\lra B}
\]
This is illustrated by the following focused derivation that introduces the formula $F$:
{\small
 \[
 \infer=[{D_\infty,2 \times \exists}]{\llfDash \lEnc \ndots{\infty}
\lft{\Gamma} \ndots{l} {\color{MyDarkBlue} \rght{\Delta, A \supset B}}
\ndots{r} \cdot
\Uparrow}
{
\infer[{\tensor}]{\llfDash \lEnc \ndots{\infty}
\lft{\Gamma} \ndots{l} {\color{MyDarkBlue} \rght{\Delta, A \supset B}}
\ndots{r} \cdot
\Downarrow
\rght{A\supset B}^\bot \tensor \nbang{l}(\nquest{l}\lft{A} \lpar
\nquest{r}\rght{B})}
{
\infer[{I}]{\llfDash \lEnc \ndots{\infty}
\lft{\Gamma} \ndots{l} {\color{MyDarkBlue} \rght{\Delta, A \supset B}}
\ndots{r} \cdot
\Downarrow
\rght{A \supset B}^\bot}{}
&
\infer[{\nbang{l}}]{\llfDash \lEnc \ndots{\infty}
\lft{\Gamma} \ndots{l} {\color{MyDarkBlue} \rght{\Delta, A \supset B}}
\ndots{r} \cdot
\Downarrow \nbang{l}(\nquest{l}\lft{A} \lpar
\nquest{r}\rght{B})}
{
\infer=[{\lpar, \nquest{r}, \nquest{l}}]{ \llfDash \lEnc \ndots{\infty}
\lft{\Gamma} \ndots{l} \cdot \ndots{r} 
\cdot \Uparrow \nquest{l}\lft{A} \lpar
\nquest{r}\rght{B})} {\llfDash
\lEnc \ndots{\infty} \lft{\Gamma, A} \ndots{l} \rght{B} \ndots{r}
\cdot \Uparrow}
}
}
}
\]
}
Here we need to explain a bit of our notation. We use two linear logic atomic predicates
$\lft{\cdot}$ and $\rght{\cdot}$ that take object-level formulas as arguments.
The former predicate specifies a formula on the left-hand side of the object
logic's sequent, while the latter a formula on the right-hand side. Thus, the collection
of atomic formulas $\lft{A_1}, \ldots \lft{A_n}, \rght{B_1}, \ldots, \rght{B_m}$
in a linear logic sequent specifies the object-level sequent $A_1, \ldots, A_n
\lra B_1, \ldots, B_m$. If $\Gamma = A_1, \ldots, A_n$ is a multiset of
formulas, then we write $\lft{\Gamma}$ for $\lft{A_1}, \ldots, \lft{A_n}$,
similarly for $\rght{\Gamma}$.

The symbols $\Uparrow$ and $\Downarrow$ are part of the sequents of the focused
system for \sell, enforcing the focusing discipline. For the derivation above,
we use three subexponential labels $\infty$, $l$ and $r$. This is captured in
the syntax by the sequent with three contexts: $\llfDash \lEnc \ndots{\infty}
\lft{A_1}, \ldots \lft{A_n}  \ndots{l} \rght{B_1}, \ldots, \rght{B_m} \ndots{r}
\cdot$ which stands for the linear logic sequent $\nquest{\infty} \lEnc,
\nquest{l} \lft{A_1}, \ldots, \nquest{l} \lft{A_n}, \nquest{r} \rght{B_1},
\ldots, \nquest{r} \rght{B_m}$, where $\lEnc$ is the theory encoding the
object-level logic.

Now, the interesting fact is that the derivation above is the only way of
introducing the formula $F$ by using the focusing discipline. Notice that this
derivation corresponds indeed to \mLJ's implication right rule: The derivation's
conclusion is $\llfDash \lEnc \ndots{\infty} \lft{\Gamma} \ndots{l}
{\rght{\Delta, A \supset B}} \ndots{r} \cdot \Uparrow$ specifying the conclusion
of \mLJ's implication right rule $\Gamma \lra \Delta, A \supset B$ and the
derivation has one open premise $\llfDash \lEnc \ndots{\infty} \lft{\Gamma, A}
\ndots{l} \rght{B} \ndots{r} \cdot \Uparrow$, which corresponds to the premise
of \mLJ's implication right rule $\Gamma, A \lra B$.

Checking whether one given linear logic formula $G$ corresponds to some
inference rule, in the same way as the formula $F$ corresponds to \mLJ's
implication right rule, requires some effort. In particular, one needs
to construct the focused derivation introducing the given formula $G$ and check whether this derivation indeed
corresponds to the desired inference rule. This is particularly challenging as a given formula may specify 
more than one rule. 

As we show in our recent work~\cite{nigam13iclp}, constructing such focused derivations is 
much easier by using propositional theories, called Answer-Sets. These theories were used for
checking which rule permutations are valid. However, one downside of this method is that it 
yields derivations which are hardly understandable for someone that is not familiar with linear logic, 
such as the focused derivation shown above. This is a serious limitation as it requires
a deep understanding of focusing in order to understand the proof figures printed.

This paper solves this problem by showing how to automatically derive and draw
the object-level inference rules
specified by a linear logic formula as they would appear in a proof theory textbook. We are currently
implementing the techniques mentioned in this paper in a tool called Quati to be released soon.
We show this procedure in a rather informal fashion, using the example above as
the running example. 
We are currently writing down the correctness proof of our specification.

% \section{Linear Logic with Subexponentials}

\section{Construction of Derivation Skeletons}

\paragraph{Derivation Skeletons}
As explained in our previous work~\cite{nigam13iclp}, we showed how to specify derivations in a declarative fashion 
by using a pair $\tup{\Xi, \Bscr}$ called \emph{derivation skeletons}, where $\Xi$ is a generic derivation and 
$\Bscr$ is a set of constraints.
Its formal definition was introduced in \cite{nigam13iclp} and we only informally describe them here.   

The set of constraints that we use are depicted in Table~\ref{fig:predicates}.
The meaning of these constraints is specified by the rules $\textrm{(r1)},
\textrm{(r2)},\ldots, \textrm{(r8)}$, expressed as logic program clauses, 
also depicted in Table~\ref{fig:predicates}. These
rules and the predicates in Table~\ref{fig:predicates} specify in a declarative 
fashion the content of a context variable, $\Gamma$, in a derivation.
The encoding is all based on atomic formulas of the form $\In{F}{\Gamma}$, which 
specify that the formula $F$ is in the context $\Gamma$. 
% As it will be clear later in this section, 
% one may specify which formulas exactly appear in the conclusion sequent of a derivation.

The atomic formula $\elin{F}{\Gamma}$ specifies that the context $\Gamma$ has a
single formula $F$. 
The first rule (r1) specifies that $\In{F}{\Gamma}$, while the second rule (r2) is a constraint
rule specifying that there is no other formula $F'$ different from $F$ in the context $\Gamma$.

%% TODO: mention the rule or change to neg_l of LJ, which requires the right
%% side of the sequent to be empty. [Giselle]
In some situations, for instance, when specifying the linear logic initial rule~\cite{girard87tcs}, 
we need to specify that some contexts are empty, which is done by using the atomic formula $\emp{\Gamma}$.  
Rule (r3) is a constraint that specifies that no formula can be in an empty context.
 
The most elaborate specification are the rules (r4) -- (r8), which specify the atomic formula 
$\union{\Gamma^1} {\Gamma^2} {\Gamma}$, i.e. $\Gamma = \Gamma^1 \cup \Gamma^2$.
The rules (r4) and (r5) specify that $\Gamma^1 \subseteq \Gamma$ and $\Gamma^2 \subseteq \Gamma$, 
that is, a formula occurrence that is in $\Gamma^i$ is also in $\Gamma$. The rule (r6) specifies that 
if both $\Gamma^1$ and $\Gamma^2$ are empty then so is $\Gamma$. 
The rules (r7) and (r8) 
specify that these contexts are bounded, that is, the union 
$\Gamma = \Gamma^1 \cup \Gamma^2$ is a multiset union. An occurrence of a formula in $\Gamma$ either comes
from $\Gamma^1$ or from $\Gamma^2$.  

\begin{table}[t]
\caption{\small List of atomic formulas used together
with their denotations and their logical axiomatization $\thSeq$. Following 
usual logic programming conventions, all non-predicate term symbols are assumed 
to be universally quantified, and we use commas, ``$,$'', for conjunctions and
 ``$\leftarrow$'' for the reverse implication.}
\label{fig:predicates}
\begin{tabular}{l@{\quad}p{2cm}@{\quad}l}
\toprule
Alphabet & Denotation & Logic Specification \\[1pt]
\midrule
$\In{F}{\Gamma}$ & $F \in \Gamma$ & No theory.\\
\midrule
$\elin{F}{\Gamma}$ &  $\Gamma = \{F\}$ & (r1) $\In{F}{\Gamma} \leftarrow \elin{F}{\Gamma}$. \\[1pt]
&& (r2) $\bot \leftarrow \In{F_1}{\Gamma}, \elin{F}{\Gamma}, F_1 \neq F$. \\
\midrule
$\emp{\Gamma}$ & $\Gamma = \emptyset$ 
&  (r3) $\bot \leftarrow \In{F}{\Gamma}, \emp{\Gamma}$. \\
\midrule
$\union{\Gamma^1} {\Gamma^2} {\Gamma}$ & $\Gamma = \Gamma^1 \cup \Gamma^2$ & 
   (r4) $\In{F}{\Gamma} \leftarrow \In{F}{\Gamma^1}, \union{\Gamma^1}{\Gamma^2}{\Gamma}$. \\[1pt]
&& (r5) $\In{F}{\Gamma} \leftarrow \In{F}{\Gamma^2}, \union{\Gamma^1}{\Gamma^2}{\Gamma}$. \\[1pt]
&& (r6) $\emp{\Gamma}  \leftarrow \emp{\Gamma^1},\emp{\Gamma^2}, \union{\Gamma^1}{\Gamma^2}{\Gamma}$. \\[1pt]
&& (r7) $\In{F}{\Gamma^1}  \leftarrow \dnot\ \In{F}{\Gamma^2}, \In{F}{\Gamma}, \union{\Gamma^1}{\Gamma^2}{\Gamma}$. \\
&& (r8) $\In{F}{\Gamma^2}  \leftarrow \dnot\ \In{F}{\Gamma^1}, \In{F}{\Gamma}, \union{\Gamma^1}{\Gamma^2}{\Gamma}$. \\
\bottomrule
\end{tabular}
\vspace{-4mm}
\end{table}

Consider the following illustrative example of how a derivation skeleton specifies 
declaratively an inference rule:

\paragraph{Example:}
Consider the $\tensor_R$ rule shown to the left. Its \emph{inference skeleton} 
is the pair $\tup{\Xi_\tensor, \Bscr_\tensor}$, where $\Xi_\tensor$ is the derivation shown to the right:
{\small
\[
\infer[\tensor_R]{{\Gamma, \Gamma'} \vdash \Delta, \Delta', A\tensor B}
{{\Gamma} \vdash \Delta, A \qquad  {\Gamma'} \vdash \Delta', B}
\qquad 
\infer{\Gamma^0_1 \vdash \Gamma^0_2 }
{
{\Gamma^1_1 \vdash \Gamma^1_2 }
&
{\Gamma^2_1 \vdash \Gamma^2_2 }
}
\]
}
Notice that the generic derivation is composed by
context variables $\Gamma^i_j$. In particular, each sequent and 
each side of the sequent is marked with a context variable, that is, its contents
are not specified, thus generic. 
It is only used for specifying the shape of the derivation and not the formulas that 
appear on it.

The configuration of formulas in the derivation are specified by set of constraints.
The set of constraints associated to the generic derivation above is:
\begin{small}
\[
\Bscr = \left\{
\begin{array}{c}
\elin{A \tensor B}{\Gamma_{aux}^1},
\elin{A}{\Gamma_{aux}^2},
\elin{B}{\Gamma_{aux}^3}, \\
\union{\Gamma_{aux}^1}{\Gamma_{aux}^4}{\Gamma^{0}_{2}},
\union{\Gamma_{aux}^2}{\Gamma_{aux}^5}{\Gamma^{1}_{2}}, 
\union{\Gamma_{aux}^3}{\Gamma_{aux}^6}{\Gamma^{2}_{2}},\\
\union{\Gamma_{aux}^5}{\Gamma_{aux}^6}{\Gamma_{aux}^4},
\union{\Gamma^1_1}{\Gamma^1_2}{\Gamma^0_1}
\end{array}
\right\}
\]
\end{small}%
Here the context variables of the form $\Gamma_{aux}^i$, where $i \in \mathbb{N}$, are 
auxiliary context variables not appearing in the generic derivation above, 
but used to specify them. It is easy to check that the Logic Program (LP) $\Bscr_\tensor \cup \Tscr$
has a single model (called answer-set by the logic programming community), 
containing the formulas $\In{A \tensor B}{\Gamma^{0}_{2}}$,
$\In{A}{\Gamma^{1}_{2}}$ and $\In{B}{\Gamma^{2}_{2}}$. There is a number of efficient 
solvers available. For Quati we use DLV~\cite{dlv}.

In our previous work~\cite{nigam13iclp}, we discuss the advantages of representing 
a derivation using logic specifications. In particular, we can reason over
these specifications, thus reasoning about different logics in a uniform way.
For instance, we developed the machinery necessary for checking whether a rule
permutes over another rule in the object logic using its \sell\ specification.
Therefore, we can check the permutation of rules in many systems using the same
method.
In the following section, we enter into the details of extracting derivation skeletons
from a linear logic specification. This is new with respect to \cite{nigam13iclp} and 
complements this work with our previous work~\cite{nigam.jlc} on encoding proof
systems in \sell. 

\subsection{Extracting a Derivation Skeleton from a Linear Logic Formula}
\label{sec:extract}

In the introduction, we briefly described that linear logic can be used as a meta-logic 
to specify object-logic inference rules with a strong level of 
adequacy~\cite{nigam10jar}. Here we will show how we can obtain the corresponding
derivation skeleton from a linear logic formula. 

We show some illustrative cases of our procedure, which is defined recursively on the 
size of formulas. Assume that there 
are $n$ subexponentials. We initialize the algorithm by first constructing a sequent:
\[
 \vdash \Gamma^{0}_{1}, \ldots, \Gamma^{0}_{n}
\]
Each context variable corresponds to one of the subexponential contexts. For example, 
for the encoding of \mLJ\ shown in the introduction, there are three contexts:
$l$, $r$, $\infty$, resulting in the following initial sequent:
\[
 \vdash \Gamma^0_{\infty}, \Gamma^{0}_{l}, \Gamma^{0}_{r}
\]
We initialize this sequent as $Seq_k$, where $k = 0$.
Now, given a linear logic formula $F$ and the sequent $Seq_k$,
our procedure runs recursively on the height of the focused derivation introducing 
the linear logic formula $F$ and returns a set of derivation skeletons all with the 
same generic derivation, $\tup{\Xi, \Bscr_1, \ldots, \Bscr_n}$. Each $\Bscr_i$ specifies
a possible derivation. 

\textbf{Case $\tensor$:} if $F$ is of the form 
$A \tensor B$, we construct the derivation skeleton:
\[
 \infer{\vdash \Gamma^{k}_{1}, \ldots, \Gamma^{k}_{n} }
 {\vdash \Gamma^{k+1}_{1}, \ldots, \Gamma^{k+1}_{n}
 \qquad \vdash \Gamma^{k+2}_{1}, \ldots, \Gamma^{k+2}_{n}}
\]
where $Seq_k$ is the sequent $\vdash \Gamma^{k}_{1}, \ldots, \Gamma^{k}_{n}$ and the context variables 
$\Gamma^{k+1}_{j}$ and $\Gamma^{k+2}_{j}$ are fresh for all $j$ that correspond to a bounded 
subexponential and $\Gamma^{k+1}_{i} = \Gamma^{k}_{i} = \Gamma^{k+2}_{i}$ for all unbounded
subexponentials. 

Let $\tup{\Xi_{k+1}, \Bscr_{k+1}^1, \ldots, \Bscr_{k+1}^m}$ and $\tup{\Xi_{k+2}, \Bscr_{k+2}^1, \ldots, \Bscr_{k+2}^l}$ be the 
sets of derivation skeletons obtained by this algorithm when using the first and second premises
with respectively $A$ and $B$. 
Then the result of the algorithm on $F$ and $Seq_k$ is the derivation skeleton with generic derivation
\[
 \infer{\vdash \Gamma^{k}_{1}, \ldots, \Gamma^{k}_{n} }
 {\Xi_{k+1}  \qquad \Xi_{k+2}}
\]
and set of set of constraints $\{\Bscr_{k+1}^i \cup \Bscr_{k+2}^j \cup \Bscr_\tensor \mid 1 \leq i \leq m, 1\leq j \leq l\}$, 
where $\Bscr_\tensor$ is 
\[
\Bscr_\tensor = \{\union{\Gamma^{k+1}_{j}}{\Gamma^{k+2}_{j}}{\Gamma^{k}_{j}} \mid
\textrm{ $j$ bounded} \}
\]
The set $\Bscr_\tensor$ simply specifies that the contents of $\Gamma^{k}_{j}$ are split among the premises if 
$j$ is a bounded subexponential.

%% TODO: maybe this case is too complicated, since the !^l rule was not
%% explained... Drop it? [Giselle]
\textbf{Case $\nbang{\ell}$:} When $F = \nbang{\ell} A$, the derivation skeleton has a single 
premise:
\[
 \infer{\vdash \Gamma^{k}_{1}, \ldots, \Gamma^{k}_{n} }
 {\vdash \Gamma^{k+1}_{1}, \ldots, \Gamma^{k+1}_{n}}
\]
where $Seq_k = \vdash \Gamma^{k}_{1}, \ldots, \Gamma^{k}_{n}$ and the context variables 
and  $\Gamma^{k+1}_{i} = \Gamma^{k}_{i}$ for all unbounded
subexponentials. 

Let $\tup{\Xi_{k+1}, \Bscr_{k+1}^1, \ldots, \Bscr_{k+1}^m }$ be the 
derivation skeletons obtained by this algorithm when using the premise
and $A$. 
The result of the algorithm on $F$ and $Seq_k$ is the derivation skeleton 
\[
 \infer{\vdash \Gamma^{k}_{1}, \ldots, \Gamma^{k}_{n} }
 {\Xi_{k+1}}
\]
and the set of constraints $\{\Bscr_{k+1}^i \cup \Bscr_{\nbang{\ell}} \mid 1 \leq i \leq m\}$, 
where $\Bscr_{\nbang{\ell}}$ is the set 
\[
\Bscr_{\nbang{\ell}} = \{\union{\Gamma^{k+1}_{j}}{\Gamma_{aux}^j}{\Gamma^{k}_{j}}, \emp{\Gamma_{aux}^j} \mid \textrm{ $j$ bounded} \} \cup 
\{\emp{\Gamma^{k+1}_{i}}\mid \textrm{ $\ell \npreceq i$} \}
\]
where all auxiliary contexts $\Gamma_{aux}^j$ are fresh. The set to the left specifies that the contents 
of $\Gamma^{k}_{i}$ are the same as the contents of $\Gamma^{k+1}_{i}$ for all contexts $i$ that are bounded and the set to the right 
specifies that all contexts $\Gamma^{k+1}_{i}$ for subexponentials $i$ such that
$\ell \npreceq i$ have to be necessarily empty. The emptiness of these contexts
is exactly the side-condition of the rule for $\nbang{\ell}$.

\textbf{Case literal:} There are two possibilities, one where $F = A^\bot$ is a literal
introduced by an initial rule or $F = A$ is an atomic formula resulting in an open premise. 
We show the former case, 
as the latter case is similar to the cases below.

This is the base case of the algorithm. It returns the derivation:
\[
 \infer{Seq_k}{}
\]
and the set of set of constraints $\{\Bscr_B^j  \mid 
\textrm{$j$ bounded} \} \cup \{\Bscr_U^j  \mid \textrm{$j$ unbounded} \}$, 
where
\[
\begin{array}{l}
 \Bscr_B^j = \{\elin{A}{\Gamma_k^j}\} \cup \{\emp{\Gamma_k^i} \mid i \neq j \textrm{ and $i$ bounded} \}\\
 \Bscr_U^j = \{\In{A}{\Gamma_k^j}\} \cup \{\emp{\Gamma_k^i} \mid \textrm{ $i$ bounded} \}\\
\end{array}
\]
The first type of set specifies the case when the matching atomic formula, $A$, appears alone in a bounded
context, while the second type of set specifies the case when the matching atomic formula appears in an 
unbounded context, in which case it does not have to appear alone.

The following example illustrates the execution of the algorithm described above.

\paragraph{Example:}
Consider the linear logic formula shown to the left that corresponds, as described in the 
introduction, to the \mLJ\ inference rule shown to the right.
\[
F = \exists A. \exists B. [\rght{A\iimp B}^\bot \tensor \nbang{l}(\nquest{l} \lft{A}
\lpar \nquest{r} \rght{B})] \qquad \qquad 
 \infer[{\iimp_R}]{\Gamma \lra A \iimp B, \Delta}{\Gamma, A
\lra B}
\]

Applying the algorithm described above, we obtain the following Derivation Skeleton, which is in fact the 
derivation obtained by our implementation, thus the difference of notation; for example $rght(\cdot)$ is used for
$\rght{\cdot}$ and $lft(\cdot)$ used for $\lft{\cdot}$:
{\small
\[
\cfrac{\cfrac{\cfrac{\cfrac{\cfrac{}
{\Gamma_{ \Gamma}^{8} ; \Gamma_{r}^{3} ; \Gamma_{l}^{3} ; \Gamma_{ infty}^{1} ;  \Downarrow \neg rght (imp (A) (B) )  :: }
\qquad
\cfrac{\cfrac{\cfrac{\cfrac{\Gamma_{ \Gamma}^{9} ; \Gamma_{r}^{6} ; \Gamma_{l}^{5} ; \Gamma_{ infty}^{1} ;  \Uparrow }
{\Gamma_{ \Gamma}^{9} ; \Gamma_{r}^{4} ; \Gamma_{l}^{5} ; \Gamma_{ infty}^{1} ;  \Uparrow  ?^{r} rght (B)  :: }}
{\Gamma_{ \Gamma}^{9} ; \Gamma_{r}^{4} ; \Gamma_{l}^{3} ; \Gamma_{ infty}^{1} ;  \Uparrow  ?^{l} lft (A)  ::  ?^{r} rght (B)  :: }}
{\Gamma_{ \Gamma}^{9} ; \Gamma_{r}^{4} ; \Gamma_{l}^{3} ; \Gamma_{ infty}^{1} ;  \Uparrow  ?^{l} lft (A)  \bindnasrepma  ?^{r} rght (B)  :: }}
{\Gamma_{ \Gamma}^{9} ; \Gamma_{r}^{3} ; \Gamma_{l}^{3} ; \Gamma_{ infty}^{1} ;  \Downarrow  !^{l}  ?^{l} lft (A)  \bindnasrepma  ?^{r} rght (B)  :: }}
{\Gamma_{ \Gamma}^{7} ; \Gamma_{r}^{3} ; \Gamma_{l}^{3} ; \Gamma_{ infty}^{1} ;  \Downarrow \neg rght (imp (A) (B) )  \otimes  !^{l}  ?^{l} lft (A)  \bindnasrepma  ?^{r} rght (B)  :: }}
{\Gamma_{ \Gamma}^{7} ; \Gamma_{r}^{3} ; \Gamma_{l}^{3} ; \Gamma_{ infty}^{1} ;  \Downarrow \exists A \neg rght (imp (A) (B) )  \otimes  !^{l}  ?^{l} lft (A)  \bindnasrepma  ?^{r} rght (B)  :: }}
{\Gamma_{ \Gamma}^{7} ; \Gamma_{r}^{3} ; \Gamma_{l}^{3} ; \Gamma_{ infty}^{1} ;  \Downarrow \exists B \exists A \neg rght (imp (A) (B) )  \otimes  !^{l}  ?^{l} lft (A)  \bindnasrepma  ?^{r} rght (B)  :: }}
% {\Gamma_{ \Gamma}^{6} ; \Gamma_{r}^{3} ; \Gamma_{l}^{3} ; \Gamma_{ infty}^{1} ;  \Uparrow }
{\Gamma_{ \Gamma}^{7} ; \Gamma_{r}^{3} ; \Gamma_{l}^{3} ; \Gamma_{ infty}^{1} ;  \Uparrow }
\]
}
%VN: I modified the derivation above. Our programs is putting the formula F in the Gamma context.
%%   I think that this should not be the case. It should be in infty.
There is just a single set of constraints attached to this generic derivation:
\[
 \left\{\begin{array}{c}
  \union{\Gamma_{ \Gamma}^{8}}{\Gamma_{ \Gamma}^{9}}{\Gamma_{ \Gamma}^{7}}, \In{rght (imp (A) (B)) }{ \Gamma_{r}^{3})}, 
  \emp{\Gamma_{ \Gamma}^{8}},\\[2pt]
  \emp{\Gamma_{ \Gamma}^{9}}, \emp{\Gamma_{r}^{4}},\elin{lft (A)}{ \Gamma_{l}^{4}},
  \elin{rght (B)}{ \Gamma_{r}^{5})},\\[2pt]
  \union{\Gamma_{l}^{3}} {\Gamma_{l}^{4}} {\Gamma_{l}^{5}},\union{\Gamma_{r}^{4}}{\Gamma_{r}^{5} } {\Gamma_{r}^{6})}.
 \end{array}\right\}
\]
Notice that the constraints specify that the context $\Gamma^6_r$ in the open premise 
may only contain the formula $\rght{B}$ as expected and moreover the formula $\rght{A \iimp B}$ is in the context
$\Gamma^3_r$. This set of constraints is given to the prover DLV together with the theory depicted in Table~\ref{fig:predicates} and the 
output of the prover are the minimal models of the theory. For this particular example, there is a single model depicted below:
\[
 \left\{\begin{array}{c}
\union{\Gamma_{\Gamma}^{8}}{\Gamma_{\Gamma}^{9}}{ \Gamma_{\Gamma}^{7}}, 
\union{\Gamma_{r}^{4}}{\Gamma_{r}^{5}}{\Gamma_{r}^{6}}, \union{\Gamma_{l}^{3}}{ \Gamma_{l}^{4}}{ \Gamma_{l}^{5}},\\[2pt]
\In{rght (imp (A) (B) )}{\Gamma_{r}^{3}}, \In{lft (A)} {\Gamma_{l}^{4}},
\In{lft (A)} {\Gamma_{l}^{5}},\In{rght (B) }{ \Gamma_{r}^{5}},\In { rght (B)} { \Gamma_{r}^{6}},\\[2pt]
\elin{ lft (A) }{ \Gamma_{l}^{4}}, \elin{rght (B)}{ \Gamma_{r}^{5}}, 
\emp{\Gamma_{\Gamma}^{7}},\emp{\Gamma_{\Gamma}^{8}},\emp{\Gamma_{\Gamma}^{9}},\emp{\Gamma_{r}^{4}}
        \end{array}
 \right\}
\]
 
It is easy to check that the algorithm described above does construct derivation
skeletons that correspond indeed to the focused derivation introducing a
formula. However, there is a downside to this representation as it is
non-standard. One needs to understand the meaning of constraints and of focused
derivations.  In the following section, we demonstrate how to transform such a
derivation skeleton into an inference rule closer to how it would be shown in a
proof theory textbook.

\section{From Derivation Skeletons to Inference Rules}

This section details how we transform a derivation skeleton $\tup{\Xi, \Bscr}$
into an inference rule. To be more precise, we transform a model $M$ of the theory $\Bscr \cup \thSeq$
into an inference rule. Thus the same derivation skeleton may have several (or no) inference rules associated 
to it, namely one inference rule to each model of its set of constraints. Given a model $M$ and the generic
derivation $\Xi$, we apply the following rewrite rules in two phases:
\[
\begin{array}{l@{\quad}r@{~}l@{\quad}r@{\quad}l@{\quad}r@{\quad}l}
\toprule
{\textrm{Phase 1:}} 
& \elin{F}{\Gamma}:  & \Gamma \rightarrow F 
& \emp{\Gamma}: & \Gamma \rightarrow \cdot 
& \union{\Gamma'}{\Gamma''}{\Gamma}: & \Gamma \rightarrow \Gamma',
\Gamma''\\
\midrule
\textrm{Phase 2:} & \multicolumn{6}{c}{
\In{F}{\Gamma}: \quad \Gamma \rightarrow \Gamma, F
}\\
\bottomrule
\end{array}
\]

%% TODO: this is a bit confusing... but I feel like I did not understand enough
%% to improve it. [Giselle]
In the first phase, we proceed as follows: We set $\Xi_i := \Xi$ where $ i := 0$.
We repeat the following procedure: If 
$\elin{F}{\Gamma}, \emp{\Gamma}, \union{\Gamma'}{\Gamma''}{\Gamma}$ are in $M$, such 
that there is an occurrence of $\Gamma$ in $\Xi_i$, then apply the corresponding 
rule, obtaining $\Xi_{i+1}$ and set $i := i + 1$. 
Let $\Xi'$ be the resulting generic derivation.
For the second phase we initialize $M_i := M$ and $\Xi_i := \Xi'$, where $i := 0$. We repeat 
the following procedure while there is a formula of the form $\In{F}{\Gamma}$ in $M_i$:
Let $\In{F}{\Gamma} \in M_i$, then we apply the corresponding rule to all occurrences of $\Gamma$ in $\Xi_i$
obtaining $\Xi_{i+1}$ and set $M_{i+1} = M \setminus \{\In{F}{\Gamma}\}$  and $i := i + 1$.

\paragraph{Example:}
For the derivation skeleton shown in the example at the end of 
Section~\ref{sec:extract} and using the model shown in that example, we obtain the following derivation:

\textbf{FIGURE MISSING}


\paragraph{Two-sided Sequents}
As we are using the one-sided presentation of linear logic with subexponentials, the derivations used 
in our derivation skeleton is one sided. This contrasts with the usual presentation of such inference
rules in textbooks that have two sided sequents. For this, however, we need more information from the 
user. In particular, the user should specify which type of formulas, of the form $\lft{\cdot}, \rght{\cdot}$ or both, 
and how many, a single or many formulas, a subexponential context is supposed to have. 
This is done in the specification by declarations like the following:
\[
 \texttt{subexpctx l many lft.} \qquad
 \texttt{subexpctx r many rght.}
\]
The declaration on the left specifies that the contexts for the subexponential
$l$ have many formulas, but all of them are of them form $\lft{\cdot}$, which
means that there are formulas to be placed to the left of the sequent, similarly
for the declaration on the right.

Given such declarations, we associate to each context of a subexponential $\Gamma_s^i$ the context
$\Gamma_s^i$, with the same name, to represent the object logic formulas in $\Gamma_s^i$ that should be 
placed to the left-hand-side of sequents, and $\Delta_s^i$ for the object logic formulas in  $\Gamma_s^i$
that should be placed to the right-hand-side. With these new contexts, the derivation above is rewritten 
to the derivation below where sequents are two-sided. The meta-level formulas of the form $\rght{F}$ (respectively, 
$\lft{F}$) are replaced by $F$ and placed to the right-hand-side (respectively, left-hand-side) of the sequent. 
We also elide contexts that have no declaration of which formulas they have, for instance, the contexts for the 
subexponential $infty$ are elided.

{\small\[\cfrac{\cfrac{\cfrac{\cfrac{\cfrac{}
{\Gamma_{l}^{3}  \vdash \Delta_{r}^{3}, imp (A) (B)  } \;\; 
\quad
\cfrac{\cfrac{\cfrac{\cfrac{\Gamma_{l}^{3}, A  \vdash B }
{\Gamma_{l}^{3}, A  \vdash  \cdot } \;\; }
{\Gamma_{l}^{3}  \vdash  \cdot } \;\; }
{\Gamma_{l}^{3}  \vdash  \cdot } \;\; }
{\Gamma_{l}^{3}  \vdash \Delta_{r}^{3}, imp (A) (B)  } \;\; }
{\Gamma_{l}^{3}  \vdash \Delta_{r}^{3}, imp (A) (B)  } \;\; }
{\Gamma_{l}^{3}  \vdash \Delta_{r}^{3}, imp (A) (B)  } \;\; }
{\Gamma_{l}^{3}  \vdash \Delta_{r}^{3}, imp (A) (B)  } \;\; }
{\Gamma_{l}^{3}  \vdash \Delta_{r}^{3}, imp (A) (B)  } \;\; \]}

Notice that, since the declaration above specifies that the subexponential $r$ contains
many formulas, but they are all placed to the right-hand-side of sequents, the derivation 
above only has the $\Delta_r^i$ contexts, while the contexts of the form $\Gamma_r^i$ are not
shown.

The last step is then to simply collapse the derivation obtained and construct the 
inference rule by using the conclusion of the derivation and its open premises. 
For the derivation above, we obtain the following inference rule:

{\small \[\cfrac{\Gamma_{l}^{3}, A  \vdash B }
{\Gamma_{l}^{3}  \vdash \Delta_{r}^{3}, imp (A) (B)  } \;\; imp_{R}\]}

Notice that such a rule is very close to how it would appear in a standard proof-theory 
textbook.

\section{Conclusions and Future Work}

This paper describes informally the procedure for extracting inference rules, as
they would appear in a proof theory textbook, from a specification of a proof system in 
subexponential linear logic. This procedure is currently being implemented
as part of a tool called Quati that computes rule permutations.

There is a number of future work directions to follow from this work. The first one is to 
formalize our ideas, part of which can be found already in our previous
work~\cite{nigam13iclp}. Besides the implementation of our method, another
direction that we are currently investigating is the extraction of \sell\
specifications from a set of inference rules, i.e. the opposite direction of
what was shown in this work. This step will facilitate the use of our tools, such as the one described
in \cite{nigam.jlc}, called TATU~\cite{tatu}, without requiring a greater
understanding of subexponential linear logic to encode proof systems in it.

Finally, we are also investigating how to use Quati in other applications, such as
the correctness of algorithms to maintain recursive distributed database~\cite{nigam12comlan}.

\bibliographystyle{plain}
\bibliography{../master}

\end{document}
