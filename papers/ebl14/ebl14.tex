\documentclass[a4paper,10pt]{article}
\usepackage[utf8]{inputenc}
\usepackage{mycommands}
\usepackage{myproof}
\usepackage{color}
\usepackage{stmaryrd}
\usepackage{latexsym}
\usepackage{url}
\usepackage{amsmath}
\usepackage{amssymb}
\usepackage{times}
% \usepackage{../mycommands}
\usepackage{multirow}
\usepackage{xspace}
\usepackage{alltt}
\usepackage{booktabs}
%\usepackage{lineno}
\usepackage{algorithm}
\usepackage{algorithmic}
\RequirePackage{txfonts}

\usepackage{xspace}
\RequirePackage{txfonts}
\definecolor{MyDarkBlue}{RGB}{0,78,126}
\definecolor{MyRed}{RGB}{255,48,48}
\newcommand\thSeq{\ensuremath{\Tscr}}
\newcommand\groundSet{\ensuremath{\mathcal{B}}}

% \newcommand\leftarrow\supset
\newcommand\lra{\longrightarrow}

% Constraint predicates
\newcommand{\elin}[2]{\ensuremath{{\tsl{unitctx}(\ensuremath{#1}, \ensuremath{#2})}}}
\newcommand{\emp}[1]{\ensuremath{{\tsl{emp}(\ensuremath{#1})}}}
\newcommand{\eqctx}[2]{\ensuremath{{\tsl{eqctx}(\ensuremath{#1},\ensuremath{#2})}}}
\newcommand{\union}[3]{\ensuremath{{\tsl{union}(\ensuremath{#1},\ensuremath{#2},\ensuremath{ #3})}}}
\newcommand{\In}[2]{\ensuremath{\tsl{in}(\ensuremath{#1},\ensuremath{#2})}}
\newcommand{\Equ}[2]{\ensuremath{\tsl{Equ}(\ensuremath{#1},\ensuremath{#2})}}
\newcommand{\unions}[2]{\ensuremath{\tsl{Unions}(\ensuremath{#1},\ensuremath{#2})}}
\newcommand{\equal}[2]{\ensuremath{\tsl{Equal}(\ensuremath{#1},\ensuremath{#2})}}
\newcommand{\equalCtx}[2]{\ensuremath{\tsl{EqualCtx}(\ensuremath{#1},\ensuremath{#2})}}
% ProveIf predicates
\newcommand{\inSequent}[2]{\ensuremath{\tsl{inSequent}(\ensuremath{#1},\ensuremath{#2})}}
\newcommand{\inDer}[2]{\ensuremath{\tsl{inDer}(\ensuremath{#1},\ensuremath{#2})}}
\newcommand{\provIf}[2]{\ensuremath{\tsl{proveIf}(\ensuremath{#1},\ensuremath{#2})}}
\newcommand{\notProvIf}[2]{\ensuremath{\tsl{notProveIf}(\ensuremath{#1},\ensuremath{#2})}}
\newcommand{\bounded}[1]{\ensuremath{{\tsl{bounded}(\ensuremath{#1})}}}

% Names of the logic programs
\newcommand\LPder{\ensuremath{\mathbb{P}_1}}
\newcommand\LPprov{\ensuremath{\mathbb{P}_2}}
\newcommand\dnot{\ensuremath{\mathit{not}}\xspace}
%opening
\title{Quati: From Linear Logic Specifications to Inference Rules (Extended Abstract)}
\author{Vivek Nigam \& Leonardo Lima \\
Universidade Federal da Paraíba, Brazil\\
\texttt{vivek.nigam@gmail.com} \& \texttt{leonardo.alfs@gmail.com} 
\and Giselle Reis\\
Technische Universit\"at Wien, Vienna, Austria}

\begin{document}

\maketitle

\begin{abstract}
In our previous work, we have shown that a great number of proof systems, such 
as a multiconclusion system for intuitionisitic logic, can 
be specified as theories in linear logic with subexponentials, a refinement of linear logic. 
These theories are natural allowing for example the development of techniques for checking
automatically non-trivial properties of the encoded systems, such as whether the encoded proof system 
admits cut-elimination or to check which rule permutations are always allowed. 
However, it takes some amount of effort to check whether a linear logic theory corresponds
indeed to a given proof system. This paper fills this gap by developing a technique that automatically 
transforms a linear logic specification into the inference rules it specifies thus allowing
the user to check for errors in the specification. We applied our technique to our previous work for 
checking and enumerating rule permutations from linear logic specifications, obtaining figures
as they would normally appear in a proof theory text book. We are currently implementing our technique 
as a tool called Quati to be launched soon.
\end{abstract}

\section{Introduction}
Quati is a tool to be launched soon that takes a specification in linear logic with subexponentials~\cite{nigam09ppdp} 
and infers automatically a number rules permutations that are always allowed. The theory and technique that 
based this tool was explained in our previous work~\cite{nigam13iclp}. This paper tackles one of 
Quati's method that was not explained before, namely, how to construct rules as they would appear in 
a standard Proof-Theory book~\cite{troelstra96bpt} from a specification in linear logic with subexponential.

Linear logic with Subexponentials is a refinement of linear logic, which allows for any number of 
exponential-like connectives, called subexponentials, written $\nbang{\ell}$ and $\nquest{\ell}$ with a label $\ell$. These labels
 are organized into a pre-order, $\prec$, specifying the provability relation among subexponentials.   
In our recent work~\cite{nigam.jlc}, we demonstrated
that a number of proof systems with rather complicated structural rules can be specified as linear logic with subexponential 
theories. For example, a multiconclusion system (\mLJ) for intuitionisitic logic or a proof system for S4 modal logic can be specified in 
Linear Logic with Subexponentials. 
In fact, our encoding is quite strong having an adequacy on the level of 
derivations~\cite{nigam10jar}. That is, there is a one-to-one correspondence between the (focused) derivations
obtained from the theory and the derivations in the encoded proof system. 

This notion of adequacy is formalized by using more advanced proof theory called focusing, for which we refrain here to 
enter into the details for readability. Further details can be found in \cite{nigam.jlc}. In a nutshell focusing
is a complete discipline for proofs (and derivations) introduced first in the context of logic programming 
by Andreoli~\cite{andreoli92jlc} for Linear 
Logic, which reduces the proof search do not know non-determinism. Focused proofs can be seen as the normal-form proofs for 
proof search. 

For our running example, we can show that the linear logic formula to the left corresponds exactly to \mLJ's implication 
right introduction rule to the right:
\[
F = \exists A. \exists B. [\rght{A\iimp B}^\bot \tensor \nbang{l}(\nquest{l} \lft{A}
\lpar \nquest{r} \rght{B})] \qquad \qquad 
 \infer[{\iimp_R}]{\Gamma \lra A \iimp B, \Delta}{\Gamma, A
\lra B}
\]
This is illustrated by the following focused derivation that introduces the formula $F$:
 \[
 \infer=[{D_\infty,2 \times \exists}]{\llfDash \lEnc \ndots{\infty}
\lft{\Gamma} \ndots{l} {\color{MyDarkBlue} \rght{\Delta, A \supset B}}
\ndots{r} \cdot
\Uparrow}
{
\infer[{\tensor}]{\llfDash \lEnc \ndots{\infty}
\lft{\Gamma} \ndots{l} {\color{MyDarkBlue} \rght{\Delta, A \supset B}}
\ndots{r} \cdot
\Downarrow
\rght{A\supset B}^\bot \tensor \nbang{l}(\nquest{l}\lft{A} \lpar
\nquest{r}\rght{B})}
{
\infer[{I}]{\llfDash \lEnc \ndots{\infty}
\lft{\Gamma} \ndots{l} {\color{MyDarkBlue} \rght{\Delta, A \supset B}}
\ndots{r} \cdot
\Downarrow
\rght{A \supset B}^\bot}{}
&
\infer[{\nbang{l}}]{\llfDash \lEnc \ndots{\infty}
\lft{\Gamma} \ndots{l} {\color{MyDarkBlue} \rght{\Delta, A \supset B}}
\ndots{r} \cdot
\Downarrow \nbang{l}(\nquest{l}\lft{A} \lpar
\nquest{r}\rght{B})}
{
\infer=[{\lpar, \nquest{r}, \nquest{l}}]{ \llfDash \lEnc \ndots{\infty}
\lft{\Gamma} \ndots{l} \cdot \ndots{r} 
\cdot \Uparrow \nquest{l}\lft{A} \lpar
\nquest{r}\rght{B})} {\llfDash
\lEnc \ndots{\infty} \lft{\Gamma, A} \ndots{l} \rght{B} \ndots{r}
\cdot \Uparrow}
}
}
}
\]
Here we need to explain a bit of our notation. We use two meta-level atomic predicates
$\lft{\cdot}$ and $\rght{\cdot}$ that takes an object-level formula to a linear logic 
one. The former predicate specifies a formula to the left-hand side of the sequent, while the latter
a formula to the right-hand side. Thus, the collection of atomic formulas $\lft{A_1}, \ldots
\lft{A_n}, \rght{B_1}, \ldots, \rght{B_m}$ in a linear logic sequent specifies the object-level
sequent $A_1, \ldots, A_n \lra B_1, \ldots, B_m$. If $\Gamma = A_1, \ldots, A_n$ is a multiset
of formulas, then we write $\lft{\Gamma}$ for $\lft{A_1}, \ldots, \lft{A_n}$, similarly for $\rght{\Gamma}$.

The derivation above with $\Uparrow$ and $\Downarrow$
are sequents of the focused system for linear logic with subexponential enforcing the focusing 
discipline. For the derivation above, we use three subexponential labels $\infty, l$
and $r$. This is captured in the syntax by the sequent with three contexts: $\llfDash \lEnc \ndots{\infty}
\lft{A_1}, \ldots \lft{A_n}  \ndots{l} \rght{B_1}, \ldots, \rght{B_m} \ndots{r} \cdot$ which stands for the linear logic sequent
$\nbang{\infty} \lEnc, \nquest{l} \lft{A_1}, \ldots, \nquest{l} \lft{A_n}, \nquest{r} \rght{B_1}, \ldots, \nquest{r} \rght{B_m}$, 
where $\lEnc$ is the theory encoding the system.

Now, the interesting fact is that the derivation above is the only way of introducing the formula $F$ 
shown above by using the focusing 
discipline. Notice that this derivation corresponds indeed to \mLJ's 
implication right rule: The derivation's conclusion is $\llfDash \lEnc \ndots{\infty}
\lft{\Gamma} \ndots{l} {\rght{\Delta, A \supset B}}
\ndots{r} \cdot \Uparrow$ specifying the conclusion of \mLJ's 
implication right rule $\Gamma \lra \Delta, A \supset B$ and 
the derivation has one premise $\llfDash \lEnc \ndots{\infty} \lft{\Gamma, A} \ndots{l} \rght{B} \ndots{r}
\cdot \Uparrow$, which corresponds to the premise of \mLJ's 
implication right rule $\Gamma, A \lra B$.

To check whether one given linear logic formula $G$ corresponds to some inference rule, in the same way as the 
the formula $F$ corresponds to \mLJ's implication right rule, requires some effort. In particular, one needs
to construct the focused derivation introducing the given formula $G$ and check whether this derivation indeed
corresponds to the desired inference rule. This is particularly challenging as a given formula may specify 
more than one rule. 

\textbf{Example here, for instance tensor.}

As we shown in our recent work~\cite{nigam13iclp}, constructing such focused derivations is 
much easier by using propositional theories, called Answer-Sets. These theories were used for
checking which rule permutations are valid. However, one downside of this method is that it 
yields derivations which are hardly understandable for someone that is familiar with linear logic, 
such as the focused derivation shown above. This is a serious limitation as it requires
a deep understanding of focusing in order to understand the proof figures printed.

This paper fills this gap by showing how to automatically derive and draw the inference rules
specified by a linear logic formula as they would appear in a proof theory textbook. We are currently
implementing the techniques mentioned in this paper in a tool called Quati to be released soon.

% \section{Linear Logic with Subexponentials}

\section{Constraints and the Construction of Derivation Skeletons}

Explain the constraints and derivation skeletons. How this is constructed using focusing.

\begin{table}[t]
\caption{\small List of atomic formulas used together
with their denotations and their logical axiomatization $\thSeq$. Following 
usual logic programming conventions, all non-predicate term symbols are assumed 
to be universally quantified, and we use commas, ``$,$'', for conjunctions and
 ``$\leftarrow$'' for the reverse implication.}
\label{fig:predicates}
\begin{tabular}{l@{\quad}p{2cm}@{\quad}l}
\toprule
Alphabet & Denotation & Logic Specification \\[1pt]
\midrule
$\In{F}{\Gamma}$ & $F \in \Gamma$ & No theory.\\
\midrule
$\elin{F}{\Gamma}$ &  $\Gamma = \{F\}$ & (r1) $\In{F}{\Gamma} \leftarrow \elin{F}{\Gamma}$. \\[1pt]
&& (r2) $\bot \leftarrow \In{F_1}{\Gamma}, \elin{F}{\Gamma}, F_1 \neq F$. \\
\midrule
$\emp{\Gamma}$ & $\Gamma = \emptyset$ 
&  (r3) $\bot \leftarrow \In{F}{\Gamma}, \emp{\Gamma}$. \\
\midrule
$\union{\Gamma^1} {\Gamma^2} {\Gamma}$ & $\Gamma = \Gamma^1 \cup \Gamma^2$ & 
   (r4) $\In{F}{\Gamma} \leftarrow \In{F}{\Gamma^1}, \union{\Gamma^1}{\Gamma^2}{\Gamma}$. \\[1pt]
&& (r5) $\In{F}{\Gamma} \leftarrow \In{F}{\Gamma^2}, \union{\Gamma^1}{\Gamma^2}{\Gamma}$. \\[1pt]
&& (r6) $\emp{\Gamma}  \leftarrow \emp{\Gamma^1},\emp{\Gamma^2}, \union{\Gamma^1}{\Gamma^2}{\Gamma}$. \\[1pt]
&& (r7) $\In{F}{\Gamma^1}  \leftarrow \dnot\ \In{F}{\Gamma^2}, \In{F}{\Gamma}, \union{\Gamma^1}{\Gamma^2}{\Gamma}$. \\
&& (r8) $\In{F}{\Gamma^2}  \leftarrow \dnot\ \In{F}{\Gamma^1}, \In{F}{\Gamma}, \union{\Gamma^1}{\Gamma^2}{\Gamma}$. \\
\bottomrule
\end{tabular}
\vspace{-4mm}
\end{table}


\section{From Derivation Skeletons to Inference Rules}

\section{Conclusions and Future Work}

\bibliographystyle{plain}
\bibliography{../master}

\end{document}
