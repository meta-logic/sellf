\documentclass[a4paper,10pt]{article}
\usepackage[utf8]{inputenc}
\usepackage{mycommands}
\usepackage{myproof}
\usepackage{color}
\usepackage{stmaryrd}
\usepackage{latexsym}
\usepackage{url}
\usepackage{amsmath}
\usepackage{amssymb}
\usepackage{times}
% \usepackage{../mycommands}
\usepackage{multirow}
\usepackage{xspace}
\usepackage{alltt}
\usepackage{booktabs}
%\usepackage{lineno}
\usepackage{algorithm}
\usepackage{algorithmic}
\RequirePackage{txfonts}
\newcommand\ctx[2]{\ensuremath{\textrm{ctx}_R(#1,#2)}}

\usepackage{xspace}
\RequirePackage{txfonts}
\definecolor{MyDarkBlue}{RGB}{0,78,126}
\definecolor{MyRed}{RGB}{255,48,48}
\newcommand\thSeq{\ensuremath{\Tscr}}
\newcommand\groundSet{\ensuremath{\mathcal{B}}}

% \newcommand\leftarrow\supset
\newcommand\lra{\longrightarrow}

% Constraint predicates
\newcommand{\elin}[2]{\ensuremath{{\tsl{unitctx}(\ensuremath{#1}, \ensuremath{#2})}}}
\newcommand{\emp}[1]{\ensuremath{{\tsl{emp}(\ensuremath{#1})}}}
\newcommand{\eqctx}[2]{\ensuremath{{\tsl{eqctx}(\ensuremath{#1},\ensuremath{#2})}}}
\newcommand{\union}[3]{\ensuremath{{\tsl{union}(\ensuremath{#1},\ensuremath{#2},\ensuremath{ #3})}}}
\newcommand{\In}[2]{\ensuremath{\tsl{in}(\ensuremath{#1},\ensuremath{#2})}}
\newcommand{\Equ}[2]{\ensuremath{\tsl{Equ}(\ensuremath{#1},\ensuremath{#2})}}
\newcommand{\unions}[2]{\ensuremath{\tsl{Unions}(\ensuremath{#1},\ensuremath{#2})}}
\newcommand{\equal}[2]{\ensuremath{\tsl{Equal}(\ensuremath{#1},\ensuremath{#2})}}
\newcommand{\equalCtx}[2]{\ensuremath{\tsl{EqualCtx}(\ensuremath{#1},\ensuremath{#2})}}
% ProveIf predicates
\newcommand{\inSequent}[2]{\ensuremath{\tsl{inSequent}(\ensuremath{#1},\ensuremath{#2})}}
\newcommand{\inDer}[2]{\ensuremath{\tsl{inDer}(\ensuremath{#1},\ensuremath{#2})}}
\newcommand{\provIf}[2]{\ensuremath{\tsl{proveIf}(\ensuremath{#1},\ensuremath{#2})}}
\newcommand{\notProvIf}[2]{\ensuremath{\tsl{notProveIf}(\ensuremath{#1},\ensuremath{#2})}}
\newcommand{\bounded}[1]{\ensuremath{{\tsl{bounded}(\ensuremath{#1})}}}

\newtheorem{theorem}{Theorem} 
\newtheorem{definition}[theorem]{Definition}
\newtheorem{lemma}[theorem]{Lemma}
\newtheorem{proposition}[theorem]{Proposition}
\newtheorem{corollary}[theorem]{Corollary}
\newenvironment{remark}{\noindent \textbf{Remark}\quad}{}
\renewcommand{\red}[1]{\textcolor{red}{#1} }
\newenvironment{Paragraph}[1]{\paragraph{#1}}


% Names of the logic programs
\newcommand\LPder{\ensuremath{\mathbb{P}_1}}
\newcommand\LPprov{\ensuremath{\mathbb{P}_2}}
\newcommand\dnot{\ensuremath{\mathit{not}}\xspace}
%opening
\title{Quati: From Linear Logic Specifications to Inference Rules (Extended Abstract)}
\author{Vivek Nigam \& Leonardo Lima \\
Universidade Federal da Paraíba, Brazil\\
\texttt{vivek.nigam@gmail.com} \& \texttt{leonardo.alfs@gmail.com} 
\and Giselle Reis\\
Technische Universit\"at Wien, Vienna, Austria}

\begin{document}

\maketitle

\begin{abstract}
In our previous work, we have shown that a great number of proof systems, such 
as a multiconclusion system for intuitionisitic logic, can 
be specified as theories in linear logic with subexponentials, a refinement of linear logic. 
These theories are natural allowing for example the development of techniques for checking
automatically non-trivial properties of the encoded systems, such as whether the encoded proof system 
admits cut-elimination or to check which rule permutations are always allowed. 
However, it takes some amount of effort to check whether a linear logic theory corresponds
indeed to a given proof system. This paper fills this gap by developing a technique that automatically 
transforms a linear logic specification into the inference rules it specifies thus allowing
the user to check for errors in the specification. We applied our technique to our previous work for 
checking and enumerating rule permutations from linear logic specifications, obtaining figures
as they would normally appear in a proof theory text book. We are currently implementing our technique 
as a tool called Quati to be launched soon.
\end{abstract}

\section{Introduction}
Quati is a tool to be launched soon that takes a specification in linear logic with subexponentials~\cite{nigam09ppdp} 
and infers automatically a number rules permutations that are always allowed. The theory and technique that 
based this tool was explained in our previous work~\cite{nigam13iclp}. This paper tackles one of 
Quati's method that was not explained before, namely, how to construct rules as they would appear in 
a standard Proof-Theory book~\cite{troelstra96bpt} from a specification in linear logic with subexponential.

Linear logic with Subexponentials is a refinement of linear logic, which allows for any number of 
exponential-like connectives, called subexponentials, written $\nbang{\ell}$ and $\nquest{\ell}$ with a label $\ell$. These labels
 are organized into a pre-order, $\prec$, specifying the provability relation among subexponentials. Moreover, some subexponentials 
 are allowed to weaken and contract, which are classified as unbounded, and the remaining subexpontntials are not allowed 
 weaken and contract, which are classified as bounded. 
In our recent work~\cite{nigam.jlc}, we demonstrated
that a number of proof systems with rather complicated structural rules can be specified as linear logic with subexponential 
theories. For example, a multiconclusion system (\mLJ) for intuitionisitic logic or a proof system for S4 modal logic can be specified in 
Linear Logic with Subexponentials. 
In fact, our encoding is quite strong having an adequacy on the level of 
derivations~\cite{nigam10jar}. That is, there is a one-to-one correspondence between the (focused) derivations
obtained from the theory and the derivations in the encoded proof system. 

This notion of adequacy is formalized by using more advanced proof theory called focusing, for which we refrain here to 
enter into the details for readability. Further details can be found in \cite{nigam.jlc}. In a nutshell focusing
is a complete discipline for proofs (and derivations) introduced first in the context of logic programming 
by Andreoli~\cite{andreoli92jlc} for Linear 
Logic, which reduces the proof search do not know non-determinism. Focused proofs can be seen as the normal-form proofs for 
proof search. 

For our running example, we can show that the linear logic formula to the left corresponds exactly to \mLJ's implication 
right introduction rule to the right:
\[
F = \exists A. \exists B. [\rght{A\iimp B}^\bot \tensor \nbang{l}(\nquest{l} \lft{A}
\lpar \nquest{r} \rght{B})] \qquad \qquad 
 \infer[{\iimp_R}]{\Gamma \lra A \iimp B, \Delta}{\Gamma, A
\lra B}
\]
This is illustrated by the following focused derivation that introduces the formula $F$:
 \[
 \infer=[{D_\infty,2 \times \exists}]{\llfDash \lEnc \ndots{\infty}
\lft{\Gamma} \ndots{l} {\color{MyDarkBlue} \rght{\Delta, A \supset B}}
\ndots{r} \cdot
\Uparrow}
{
\infer[{\tensor}]{\llfDash \lEnc \ndots{\infty}
\lft{\Gamma} \ndots{l} {\color{MyDarkBlue} \rght{\Delta, A \supset B}}
\ndots{r} \cdot
\Downarrow
\rght{A\supset B}^\bot \tensor \nbang{l}(\nquest{l}\lft{A} \lpar
\nquest{r}\rght{B})}
{
\infer[{I}]{\llfDash \lEnc \ndots{\infty}
\lft{\Gamma} \ndots{l} {\color{MyDarkBlue} \rght{\Delta, A \supset B}}
\ndots{r} \cdot
\Downarrow
\rght{A \supset B}^\bot}{}
&
\infer[{\nbang{l}}]{\llfDash \lEnc \ndots{\infty}
\lft{\Gamma} \ndots{l} {\color{MyDarkBlue} \rght{\Delta, A \supset B}}
\ndots{r} \cdot
\Downarrow \nbang{l}(\nquest{l}\lft{A} \lpar
\nquest{r}\rght{B})}
{
\infer=[{\lpar, \nquest{r}, \nquest{l}}]{ \llfDash \lEnc \ndots{\infty}
\lft{\Gamma} \ndots{l} \cdot \ndots{r} 
\cdot \Uparrow \nquest{l}\lft{A} \lpar
\nquest{r}\rght{B})} {\llfDash
\lEnc \ndots{\infty} \lft{\Gamma, A} \ndots{l} \rght{B} \ndots{r}
\cdot \Uparrow}
}
}
}
\]
Here we need to explain a bit of our notation. We use two meta-level atomic predicates
$\lft{\cdot}$ and $\rght{\cdot}$ that takes an object-level formula to a linear logic 
one. The former predicate specifies a formula to the left-hand side of the sequent, while the latter
a formula to the right-hand side. Thus, the collection of atomic formulas $\lft{A_1}, \ldots
\lft{A_n}, \rght{B_1}, \ldots, \rght{B_m}$ in a linear logic sequent specifies the object-level
sequent $A_1, \ldots, A_n \lra B_1, \ldots, B_m$. If $\Gamma = A_1, \ldots, A_n$ is a multiset
of formulas, then we write $\lft{\Gamma}$ for $\lft{A_1}, \ldots, \lft{A_n}$, similarly for $\rght{\Gamma}$.

The derivation above with $\Uparrow$ and $\Downarrow$
are sequents of the focused system for linear logic with subexponential enforcing the focusing 
discipline. For the derivation above, we use three subexponential labels $\infty, l$
and $r$. This is captured in the syntax by the sequent with three contexts: $\llfDash \lEnc \ndots{\infty}
\lft{A_1}, \ldots \lft{A_n}  \ndots{l} \rght{B_1}, \ldots, \rght{B_m} \ndots{r} \cdot$ which stands for the linear logic sequent
$\nquest{\infty} \lEnc, \nquest{l} \lft{A_1}, \ldots, \nquest{l} \lft{A_n}, \nquest{r} \rght{B_1}, \ldots, \nquest{r} \rght{B_m}$, 
where $\lEnc$ is the theory encoding the system.

Now, the interesting fact is that the derivation above is the only way of introducing the formula $F$ 
shown above by using the focusing 
discipline. Notice that this derivation corresponds indeed to \mLJ's 
implication right rule: The derivation's conclusion is $\llfDash \lEnc \ndots{\infty}
\lft{\Gamma} \ndots{l} {\rght{\Delta, A \supset B}}
\ndots{r} \cdot \Uparrow$ specifying the conclusion of \mLJ's 
implication right rule $\Gamma \lra \Delta, A \supset B$ and 
the derivation has one premise $\llfDash \lEnc \ndots{\infty} \lft{\Gamma, A} \ndots{l} \rght{B} \ndots{r}
\cdot \Uparrow$, which corresponds to the premise of \mLJ's 
implication right rule $\Gamma, A \lra B$.

To check whether one given linear logic formula $G$ corresponds to some inference rule, in the same way as the 
the formula $F$ corresponds to \mLJ's implication right rule, requires some effort. In particular, one needs
to construct the focused derivation introducing the given formula $G$ and check whether this derivation indeed
corresponds to the desired inference rule. This is particularly challenging as a given formula may specify 
more than one rule. 

As we shown in our recent work~\cite{nigam13iclp}, constructing such focused derivations is 
much easier by using propositional theories, called Answer-Sets. These theories were used for
checking which rule permutations are valid. However, one downside of this method is that it 
yields derivations which are hardly understandable for someone that is familiar with linear logic, 
such as the focused derivation shown above. This is a serious limitation as it requires
a deep understanding of focusing in order to understand the proof figures printed.

This paper fills this gap by showing how to automatically derive and draw the inference rules
specified by a linear logic formula as they would appear in a proof theory textbook. We are currently
implementing the techniques mentioned in this paper in a tool called Quati to be released soon.

% \section{Linear Logic with Subexponentials}

\section{Construction of Derivation Skeletons}

\paragraph{Derivation Skeletons}
As explained in our previous work~\cite{nigam13iclp}, we showed how to specify derivations in a declarative fashion 
by using a pair $\tup{\Xi, \Bscr}$ called \emph{derivation skeletons}, where $\Xi$ is a generic derivation and 
$\Bscr$ is a set of constraints.
Its formal definition was introduced in \cite{nigam13iclp} and we only informally describe them here.   

The set of constraints that we use are depicted in Table~\ref{fig:predicates}.
The meaning of these constraints are specified by the logic formulas, expressed as logic program clauses, 
also depicted in Table~\ref{fig:predicates}, with 
only eight rules: $\textrm{(r1)}, \textrm{(r2)},\ldots, \textrm{(r8)}$. These
rules and the predicates in Table~\ref{fig:predicates} specify in a declarative 
fashion the content of a context variable, $\Gamma$, in a derivation.
The encoding is all based on atomic formulas of the form $\In{F}{\Gamma}$, which 
specify that the formula $F$ is in the context $\Gamma$. 
% As it will be clear later in this section, 
% one may specify which formulas exactly appear in the conclusion sequent of a derivation.

The atomic formula $\elin{F}{\Gamma}$ specifies that the context $\Gamma$ has a
single formula $F$. 
The first rule (r1) specifies that $\In{F}{\Gamma}$, while the second rule (r2) is a constraint
rule specifying that there is no other formula $F'$ different from $F$ in the context $\Gamma$.

In some situations, for instance, when specifying the linear logic initial rule~\cite{girard87tcs}, 
we need to specify that some contexts are empty, which is done by using the atomic formula $\emp{\Gamma}$.  
Rule (r3) is a constraint that specifies that no formula can be in an empty context.
 
The most elaborate specification are the rules (r4) -- (r8), which specify the atomic formula 
$\union{\Gamma^1} {\Gamma^2} {\Gamma}$, i.e. $\Gamma = \Gamma^1 \cup \Gamma^2$.
The rules (r4) and (r5) specify that $\Gamma^1 \subseteq \Gamma$ and $\Gamma^2 \subseteq \Gamma$, 
that is, the occurrence of a formula that is in $\Gamma^i$ is also in $\Gamma$. The rule (r6) specifies that 
if both $\Gamma^1$ and $\Gamma^2$ are empty then so is $\Gamma$. 
The rules (r7) and (r8) 
specify that these contexts are bounded, that is, the union 
$\Gamma = \Gamma^1 \cup \Gamma^2$ is a multiset union. An occurrence of a formula in $\Gamma$ either comes
from $\Gamma^1$ or from $\Gamma^2$.  

\begin{table}[t]
\caption{\small List of atomic formulas used together
with their denotations and their logical axiomatization $\thSeq$. Following 
usual logic programming conventions, all non-predicate term symbols are assumed 
to be universally quantified, and we use commas, ``$,$'', for conjunctions and
 ``$\leftarrow$'' for the reverse implication.}
\label{fig:predicates}
\begin{tabular}{l@{\quad}p{2cm}@{\quad}l}
\toprule
Alphabet & Denotation & Logic Specification \\[1pt]
\midrule
$\In{F}{\Gamma}$ & $F \in \Gamma$ & No theory.\\
\midrule
$\elin{F}{\Gamma}$ &  $\Gamma = \{F\}$ & (r1) $\In{F}{\Gamma} \leftarrow \elin{F}{\Gamma}$. \\[1pt]
&& (r2) $\bot \leftarrow \In{F_1}{\Gamma}, \elin{F}{\Gamma}, F_1 \neq F$. \\
\midrule
$\emp{\Gamma}$ & $\Gamma = \emptyset$ 
&  (r3) $\bot \leftarrow \In{F}{\Gamma}, \emp{\Gamma}$. \\
\midrule
$\union{\Gamma^1} {\Gamma^2} {\Gamma}$ & $\Gamma = \Gamma^1 \cup \Gamma^2$ & 
   (r4) $\In{F}{\Gamma} \leftarrow \In{F}{\Gamma^1}, \union{\Gamma^1}{\Gamma^2}{\Gamma}$. \\[1pt]
&& (r5) $\In{F}{\Gamma} \leftarrow \In{F}{\Gamma^2}, \union{\Gamma^1}{\Gamma^2}{\Gamma}$. \\[1pt]
&& (r6) $\emp{\Gamma}  \leftarrow \emp{\Gamma^1},\emp{\Gamma^2}, \union{\Gamma^1}{\Gamma^2}{\Gamma}$. \\[1pt]
&& (r7) $\In{F}{\Gamma^1}  \leftarrow \dnot\ \In{F}{\Gamma^2}, \In{F}{\Gamma}, \union{\Gamma^1}{\Gamma^2}{\Gamma}$. \\
&& (r8) $\In{F}{\Gamma^2}  \leftarrow \dnot\ \In{F}{\Gamma^1}, \In{F}{\Gamma}, \union{\Gamma^1}{\Gamma^2}{\Gamma}$. \\
\bottomrule
\end{tabular}
\vspace{-4mm}
\end{table}

Consider the following illustrative example of how an inference derivation specifies 
declaratively an inference rule:

\paragraph{Example:}
Consider the $\tensor_R$ rule shown to the left. The \emph{inference skeleton} for 
it is the pair $\tup{\Xi_\tensor, \Bscr_\tensor}$, where $\Xi_\tensor$ is the derivation shown to the right:
{\small
\[
\infer[\tensor_R]{{\Gamma, \Gamma'} \vdash \Delta, \Delta', A\tensor B}
{{\Gamma} \vdash \Delta, A \qquad  {\Gamma'} \vdash \Delta', B}
\qquad 
\infer{\Gamma_{0,1} \vdash \Gamma_{0,2} }
{
{\Gamma_{1,1} \vdash \Gamma_{1,2} }
&
{\Gamma_{2,1} \vdash \Gamma_{2,2} }
}
\]
}
Notice that the generic derivation is composed by
context variables, written of the $\Gamma_{i,j}$. In particular, each sequent and 
each side of the sequent is marked with context variable, that is, where its contents
are not specified, thus generic. 
It is only used for specifying the shape of the derivation and not the formulas that 
appear on it.

The configuration of formulas in the derivation are specified by set of constraints.
For example, the set of constraints associated to the generic derivation above is the 
set:
\begin{small}
\[
\Bscr = \left\{
\begin{array}{c}
\elin{A \tensor B}{\Gamma_{aux}^1},
\elin{A}{\Gamma_{aux}^2},
\elin{B}{\Gamma_{aux}^3}, \\
\union{\Gamma_{aux}^1}{\Gamma_{aux}^4}{\Gamma_{0,2}},
\union{\Gamma_{aux}^2}{\Gamma_{aux}^5}{\Gamma_{1,2}}, 
\union{\Gamma_{aux}^3}{\Gamma_{aux}^6}{\Gamma_{2,2}},\\
\union{\Gamma_{aux}^5}{\Gamma_{aux}^6}{\Gamma_{aux}^4},
\union{\Gamma_{1,1}}{\Gamma_{1,2}}{\Gamma_{0,1}}
\end{array}
\right\}
\]
\end{small}%
Here the context variables of the form $\Gamma_{aux}^i$, where $i \in \mathbb{N}$, are 
auxiliary context variables not appearing in the generic derivation above, 
but used to specify them. It is easy to check that the Logic Program (LP) $\Bscr_\tensor \cup \Tscr$
has a single model (called answer-set by the logic programming community), 
containing the formulas $\In{A \tensor B}{\Gamma_{0,2}}$,
$\In{A}{\Gamma_{1,2}}$ and $\In{B}{\Gamma_{2,2}}$. There is a number of efficient 
solvers available. For Quati we have used DLV~\cite{dlv}.

In our previous work~\cite{nigam13iclp}, we discuss the advantages of representing 
a derivation by using logic specifications. For instance, we developed the machinery 
necessary for checking whether a rule permutes over another rule in a given proof system.
In the following section, we enter into the details of extracting derivation skeletons
from a lineaer logic specification. This is new with respect to \cite{nigam13iclp} and 
complements this work with our previous work~\cite{nigam.jlc} on encoding proof systems with linear logic with 
subexponentials 

\subsection{Extracting a Derivation Skeleton from a Linear Logic Formula}

In the Introduction, we briefly described that linear logic can be used as a meta-logic 
to specify in a strong fashion object-logic inference rules with a strong level of 
adequacy~\cite{nigam10jar}. Here we will show how we can obtain the corresponding
Derivation Skeleton from a Linear Logic Formula. 

We show some illustrative cases. Assume that there 
are $n$ subexponentials. We initialize the algorithm by first constructing a sequent:
\[
 \vdash \Gamma_{0,1}, \ldots, \Gamma_{0,n}
\]
Each context variable corresponds to one of the subexponential context. For example, 
for the encoding of \mLJ\ shown in the Introduction, there are three context $l,r, \infty$
resulting in the following initial sequent:
\[
 \vdash \Gamma_{0,\infty}, \Gamma_{0,l}, \Gamma_{0,r}
\]
We initialize this sequent as $Seq_k$, where $k = 0$.
Now, given a linear logic formula $F$ and the sequent $Seq_k$,
our procedure runs recursively  on the height of the focused derivation introducing 
the linear logic formula $F$ and returns a set of derivation skeletons all with the 
same generic derivation, $\tup{\Xi, \Bscr_1, \ldots, \Bscr_n}$. Each $\Bscr_i$ specifies
a possible derivation. 

\textbf{Case $\tensor$:} if $F$ is of the form 
$A \tensor B$, we construct the derivation skeleton:
\[
 \infer{\vdash \Gamma_{k,1}, \ldots, \Gamma_{k, n} }
 {\vdash \Gamma_{k+1,1}, \ldots, \Gamma_{k+1, n}
 \qquad \vdash \Gamma_{k+2,1}, \ldots, \Gamma_{k+2, n}}
\]
where $Seq_k$ is the sequent $\vdash \Gamma_{k,1}, \ldots, \Gamma_{k, n}$ and the context variables 
$\Gamma_{k+1, j}$ and $\Gamma_{k+2, j}$ are fresh for all $j$ that correspond to a bounded 
subexponential and $\Gamma_{k+1, i} = \Gamma_{k, i} = \Gamma_{k+2, i}$ for all unbounded
subexponentials. 

Let $\tup{\Xi_{k+1}, \Bscr_{k+1}^1, \ldots, \Bscr_{k+1}^m}$ and $\tup{\Xi_{k+2}, \Bscr_{k+2}^1, \ldots, \Bscr_{k+2}^l}$ be the 
sets of derivation skeletons obtained by this algorithm when using the first and second premises
with respectively $A$ and $B$. 
Then the result of the algorithm on $F$ and $Seq_k$ is the derivation skeleton with generic derivation
\[
 \infer{\vdash \Gamma_{k,1}, \ldots, \Gamma_{k, n} }
 {\Xi_{k+1}  \qquad \Xi_{k+2}}
\]
and set of set of constraints $\{\Bscr_{k+1}^i \cup \Bscr_{k+2}^j \cup \Bscr_\tensor \mid 1 \leq i \leq m, 1\leq j \leq l\}$, 
where $\Bscr_\tensor$ is 
\[
\Bscr_\tensor = \{\union{\Gamma_{k+1, j}}{\Gamma_{k+2, j}}{\Gamma_{k, j}} \mid \textrm{ $k$ bounded} \}
\]
The set $\Bscr_\tensor$ simply specifies that the contents of $\Gamma_{k, j}$ are split among the premises if 
$k$ is a bounded subexponential.


\textbf{Case $\nbang{\ell}$:} When $F = \nbang{\ell} A$, the derivation skeleton has a single 
premise:
\[
 \infer{\vdash \Gamma_{k,1}, \ldots, \Gamma_{k, n} }
 {\vdash \Gamma_{k+1,1}, \ldots, \Gamma_{k+1, n}}
\]
where $Seq_k = \vdash \Gamma_{k,1}, \ldots, \Gamma_{k, n}$ and the context variables 
and  $\Gamma_{k+1, i} = \Gamma_{k, i}$ for all unbounded
subexponentials. 

Let $\tup{\Xi_{k+1}, \Bscr_{k+1}^1, \ldots, \Bscr_{k+1}^m }$ be the 
derivation skeletons obtained by this algorithm when using the premise
and $A$. 
The result of the algorithm on $F$ and $Seq_k$ is the derivation skeleton 
\[
 \infer{\vdash \Gamma_{k,1}, \ldots, \Gamma_{k, n} }
 {\Xi_{k+1}}
\]
and the set of constraints $\{\Bscr_{k+1}^i \cup \Bscr_{\nbang{\ell}} \mid 1 \leq i \leq m\}$, 
where $\Bscr_{\nbang{\ell}}$ is the set 
\[
\Bscr_{\nbang{\ell}} = \{\union{\Gamma_{k+1, j}}{\Gamma_{aux}^j}{\Gamma_{k, j}}, \emp{\Gamma_{aux}^j} \mid \textrm{ $j$ bounded} \} \cup 
\{\emp{\Gamma_{k+1, i}}\mid \textrm{ $\ell \npreceq i$} \}
\]
where all auxiliary contexts $\Gamma_{aux}^j$ are fresh. The set to the left specifies that the contents 
of $\Gamma_{k, i}$ are the same as the contents of $\Gamma_{k+1, i}$ for all contexts $i$ that are bounded and the set to the right 
specifies that all contexts $\Gamma_{k+1, i}$ for subexponentials $i$ such that $\ell \npreceq i$ have to be necessarily empty.

\textbf{Case literal:} There are two possibilities, one where Let $F = A^\bot$ is a literal
introduced by an initial rule or $F = A$ is an atomic formula resulting in an open premise. 
We show the former case, 
as the latter case is similar to the cases below.

This is the base case of the algorithm. It returns the derivation:
\[
 \infer{Seq_k}{}
\]
and the set of set of constraints $\{\Bscr_B^j  \mid 
\textrm{$j$ bounded} \} \cup \{\Bscr_U^j  \mid \textrm{$j$ unbounded} \}$, 
where
\[
\begin{array}{l}
 \Bscr_B^j = \{\elin{A}{\Gamma_k^j}\} \cup \{\emp{\Gamma_k^i} \mid i \neq j \textrm{ and $i$ bounded} \}\\
 \Bscr_U^j = \{\In{A}{\Gamma_k^j}\} \cup \{\emp{\Gamma_k^i} \mid i \neq j \textrm{ and $i$ bounded} \}\\
\end{array}
\]
The first type of set specifies the case when the matching atomic formula, $A$, appears alone in a bounded
context, while the second type of set specifies the case when the matching atomic formula appears in an 
unbounded context, in which case it does not have to appear alone.

\textbf{Example:}


\[
F = \exists A. \exists B. [\rght{A\iimp B}^\bot \tensor \nbang{l}(\nquest{l} \lft{A}
\lpar \nquest{r} \rght{B})] \qquad \qquad 
 \infer[{\iimp_R}]{\Gamma \lra A \iimp B, \Delta}{\Gamma, A
\lra B}
\]


{\small
\[
\cfrac{\cfrac{\cfrac{\cfrac{\cfrac{}
{\Gamma_{ \Gamma}^{8} ; \Gamma_{r}^{3} ; \Gamma_{l}^{3} ; \Gamma_{ infty}^{1} ;  \Downarrow \neg rght (imp (A) (B) )  :: }
\cfrac{\cfrac{\cfrac{\cfrac{\Gamma_{ \Gamma}^{9} ; \Gamma_{r}^{6} ; \Gamma_{l}^{5} ; \Gamma_{ infty}^{1} ;  \Uparrow }
{\Gamma_{ \Gamma}^{9} ; \Gamma_{r}^{4} ; \Gamma_{l}^{5} ; \Gamma_{ infty}^{1} ;  \Uparrow  ?^{r} rght (B)  :: }}
{\Gamma_{ \Gamma}^{9} ; \Gamma_{r}^{4} ; \Gamma_{l}^{3} ; \Gamma_{ infty}^{1} ;  \Uparrow  ?^{l} lft (A)  ::  ?^{r} rght (B)  :: }}
{\Gamma_{ \Gamma}^{9} ; \Gamma_{r}^{4} ; \Gamma_{l}^{3} ; \Gamma_{ infty}^{1} ;  \Uparrow  ?^{l} lft (A)  \bindnasrepma  ?^{r} rght (B)  :: }}
{\Gamma_{ \Gamma}^{9} ; \Gamma_{r}^{3} ; \Gamma_{l}^{3} ; \Gamma_{ infty}^{1} ;  \Downarrow  !^{l}  ?^{l} lft (A)  \bindnasrepma  ?^{r} rght (B)  :: }}
{\Gamma_{ \Gamma}^{7} ; \Gamma_{r}^{3} ; \Gamma_{l}^{3} ; \Gamma_{ infty}^{1} ;  \Downarrow \neg rght (imp (A) (B) )  \otimes  !^{l}  ?^{l} lft (A)  \bindnasrepma  ?^{r} rght (B)  :: }}
{\Gamma_{ \Gamma}^{7} ; \Gamma_{r}^{3} ; \Gamma_{l}^{3} ; \Gamma_{ infty}^{1} ;  \Downarrow \exists A \neg rght (imp (A) (B) )  \otimes  !^{l}  ?^{l} lft (A)  \bindnasrepma  ?^{r} rght (B)  :: }}
{\Gamma_{ \Gamma}^{7} ; \Gamma_{r}^{3} ; \Gamma_{l}^{3} ; \Gamma_{ infty}^{1} ;  \Downarrow \exists B \exists A \neg rght (imp (A) (B) )  \otimes  !^{l}  ?^{l} lft (A)  \bindnasrepma  ?^{r} rght (B)  :: }}
{\Gamma_{ \Gamma}^{6} ; \Gamma_{r}^{3} ; \Gamma_{l}^{3} ; \Gamma_{ infty}^{1} ;  \Uparrow }
\]
}
 $\union{\Gamma_{ \Gamma}^{8}}{\Gamma_{ \Gamma}^{9}}{\Gamma_{ \Gamma}^{7}}$, $\In{rght (imp (A) (B)) }{ \Gamma_{r}^{3})}$, $\emp{\Gamma_{ \Gamma}^{8}}$, $\emp{\Gamma_{ \Gamma}^{9}}$, $\emp{\Gamma_{r}^{4})}$, $\elin{lft (A)}{ \Gamma_{l}^{4}}$, $\union{\Gamma_{l}^{3}} {\Gamma_{l}^{4}} {\Gamma_{l}^{5}}$, $\elin{rght (B)}{ \Gamma_{r}^{5})}$, $\union{\Gamma_{r}^{4}}{\Gamma_{r}^{5} } {\Gamma_{r}^{6})}.$





\section{From Derivation Skeletons to Inference Rules}

\section{Conclusions and Future Work}

\bibliographystyle{plain}
\bibliography{../master}

\end{document}
