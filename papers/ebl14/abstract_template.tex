\documentclass{ebl}

% Use this command to insert the title of your work.
\title{Usage Example}

% Use this command to insert authors names and identify the affiliations.
% After each author insert a \inst{} with a number to identify the institution.
% Authors with the same affiliation must have the same number.
\author{Bruno Lopes\inst{1} \and Petrucio Viana\inst{2} \and Renata Freitas\inst{2}}

% Use this command to insert the affiliation details.
% To break lines use \\
% After an institute if you need to insert an other institute use the command \nextinstitute inside the \institute{}.
% To insert the email of each author use the command \email{} inside the \institute{}.
\institute{Departamento de Inform\'atica\\
Pontif\'\i cia Univerisidade Cat\'olica do Rio de Janeiro
\email{bvieira@inf.puc-rio.br}
\nextinstitute
Departamento de An\'alise\\
Instituto de Matem\'atica e Estat\'\i stica\\
Universidade Federal Fluminense
\email{petrucio@cos.ufrj.br}
\email{renatafreitas@id.uff.br}
}

% To insert an acknowledgement use this command.
% If you would not like to have acknowledgements, please remove the following line.
\acknowledgement{The author thanks to\dots}

\begin{document}

% This command is mandatory to generate the headers of your work.
\maketitle

%Here you can insert the text of your work,
This is an usage example of EBL proceedings class.
Citations may be contextualised as~\citet{Prawitz2006} or between parenthesises as~\citep{vanDalen2008}.

% Use this command if you need bibliographic references.
% You must replace ref by the name of your BibTeX file which contains the information of your references.
% To construct your BibTeX file you should use JabRef, a manager for BibTeX entries.
% JabRef is freely available at http://jabref.sourceforge.net.
\bibliography{ref}

\end{document}