\documentclass[a4paper, 11pt]{article}
\usepackage[utf8]{inputenc}
\usepackage{amsfonts}
\usepackage{amssymb}
\usepackage{amsmath}
\usepackage{stmaryrd}
\usepackage{proof}
\usepackage{color}
\usepackage{url}

\title{Informal system description}
\author{Giselle Reis}

\begin{document}
\maketitle

A quick diagram of what is implemented so far (and available at
\url{http://www.logic.at/people/giselle/tatu}) is the following:

\vspace{1cm}

\begin{tabular}{|p{1.7cm}|}
\hline
Sequent calculus\\
\hline
\end{tabular}
$\overset{(a)}{\Longrightarrow}$
\begin{tabular}{|p{2.3cm}|}
\hline
SELLF specification\\
\hline
\end{tabular}
$\Longrightarrow$
\begin{tabular}{|p{1.5cm}|}
\hline
\textsf{\textbf{TATU}}\\
\hline
\end{tabular}
$
\left\{
  \begin{array}{l}
  \text{\scriptsize{Principal cut (1)}}\\
  \text{\scriptsize{Atomic cut elimination (2)}}\\
  \text{\scriptsize{Cut-coherence (3)}}\\
  \text{\scriptsize{Initial coherence (4)}}\\
  \end{array}
\right.
$

\vspace{1cm}

Quick explanation of each functionality:

\begin{enumerate}
  \item \textbf{Principal cut}: Checks if cuts and rules can be permuted in such
  a way that cuts become principals. This is done by checking which subexponential
  indices are used in the specification of the rules (cut and introduction
  rules).
  
  \item \textbf{Atomic cut elimination}: Checks whether atomic cuts can be
  eliminated. This is also done by checking which subexponentials indices are
  used in the specification of the rules (cut and introduction rules).
  
  \item \textbf{Cut-coherence}: Checks whether cuts on some formula with
  $\diamond$ as its top-level connective can be replaced by simpler cuts. This
  is done by checking the duality of the specifications for the introduction
  rules of $\diamond$. Let $B_r$ and $B_l$ be the specification of the right and
  left introduction rules for $\diamond$, respectively. Then we must prove: $\vdash Cut,
  B_r^{\perp} \bindnasrepma B_l^{\perp}$.

  \item \textbf{Initial coherence}: Checks wheter initial rules on some formula
  with $\diamond$ as its top-level connective can be replaced by simpler initial
  rules. Again, this is done by checking some kind of duality of the formulas.
  Let $B_r$ and $B_l$ be the specification of the right and left introduction
  rules for $\diamond$, respectively. Then we must prove $\vdash Init,
  ?^{\infty} B_r \bindnasrepma ?^{\infty} B_l$.
\end{enumerate}

For numbers 3 and 4, a bounded proof seach procedure was implemented (the bound
is set to 4 by default).

In order to implement the proof search procedure, the SELLF system was adapted with
input and output contexts, following the same ideas of ``Efficient Resource
Management for Linear Logic Proof Search'', by Cervesato, Hodas and Pfenning.
This is the system in Figure \ref{figure:sellf}. These contexts are used to
avoid the non-determinism of splitting the contexts in the $\otimes$-rule.

Axioms set the output context. This should be empty when the procedure returns
to the end sequent. In-contexts are set bottom up and out-contexts are set
top down (when returned from the recursive call).

{\color{red}Explain initial rule, top, one and bang and tensor.}


\begin{figure}[t]
{\small
$$
\infer[\text{[$\binampersand$]}]{\vdash \mathcal{K}^I / \mathcal{K}^O : \Gamma^I
/ \Gamma^O \Uparrow L, F \binampersand G}{\vdash \mathcal{K}^I / \mathcal{K}^O :
\Gamma^I / \Gamma^O \Uparrow L, F & \vdash \mathcal{K}^I / \mathcal{K}^O :
\Gamma^I / \Gamma^O \Uparrow L, G}
\qquad
\infer[\text{[$\bindnasrepma$]}]{\vdash \mathcal{K}^I / \mathcal{K}^O : \Gamma^I
/ \Gamma^O \Uparrow L, F \bindnasrepma G}{\vdash \mathcal{K}^I / \mathcal{K}^O :
\Gamma^I / \Gamma^O \Uparrow L, F, G}
$$
\vspace{-2.5mm}
$$
\infer[\text{[$\top$]}]{\vdash \mathcal{K} \otimes \mathcal{K}^O / \mathcal{K}^O:
\Gamma, \Gamma^O / \Gamma^O \Uparrow L, \top}{}
\qquad
\infer[\text{[$\bot$]}]{\vdash \mathcal{K}^I / \mathcal{K}^O : \Gamma^I /
\Gamma^O \Uparrow L, \bot}{\vdash \mathcal{K}^I / \mathcal{K}^O : \Gamma^I /
\Gamma^O \Uparrow L}
$$
\vspace{-2.5mm}
$$
\infer[\text{[$\forall$]}]{\vdash \mathcal{K}^I / \mathcal{K}^O : \Gamma^I /
\Gamma^O \Uparrow L, \forall x.F}{\vdash \mathcal{K}^I / \mathcal{K}^O :
\Gamma^I / \Gamma^O \Uparrow L, F\{c/x\}}
\qquad 
\infer[\text{[$?^l$]}]{\vdash \mathcal{K}^I / \mathcal{K}^O : \Gamma^I /
\Gamma^O \Uparrow L, ?^l F}{\vdash \mathcal{K}^I +_l F / \mathcal{K}^O :
\Gamma^I / \Gamma^O \Uparrow L}
$$
\vspace{-2.5mm}
$$
\infer[\text{[$\oplus_i$]}]{\vdash \mathcal{K}^I / \mathcal{K}^O : \Gamma^I /
\Gamma^O \Downarrow F_1 \oplus F_2}{\vdash \mathcal{K}^I / \mathcal{K}^O :
\Gamma^I / \Gamma^O \Downarrow F_i}
\qquad
\infer[\text{[$\otimes$]}]{\vdash \mathcal{K}^I / \mathcal{K}^O : \Gamma^I /
\Gamma^O \Downarrow F \otimes G}{\vdash \mathcal{K}^I / \mathcal{K}' : \Gamma^I
/ \Gamma' \Downarrow F & \vdash \mathcal{K}' / \mathcal{K}^O : \Gamma' /
\Gamma^O \Downarrow G}
$$
\vspace{-2.5mm}
$$
\infer[\text{[1]}]{\vdash \mathcal{K}^I / \mathcal{K}^I : \Gamma^I / \Gamma^I
\Downarrow 1}{}
\qquad
\infer[\text{[$\exists$]}]{\vdash \mathcal{K}^I / \mathcal{K}^O : \Gamma^I /
\Gamma^O \Downarrow \exists x.F}{\vdash \mathcal{K}^I / \mathcal{K}^O : \Gamma^I
/ \Gamma^O \Downarrow F\{t/x\}}
$$
\vspace{-2.5mm}
$$
\infer[\text{[$!^l$]}]{\vdash \mathcal{K}^I / \mathcal{K}^I >_l, \mathcal{K}^O :
\Gamma^I / \Gamma^I \Downarrow !^l F}{\vdash \mathcal{K}^I \leq_l /
\mathcal{K}^O : \cdot \Uparrow F}
$$
\vspace{-2.5mm}
$$
\infer[\text{[I, given $A \in (\Gamma \cup
\mathcal{K}^I)$]}]{\vdash \mathcal{K}^I / \mathcal{K}^I - \{A\} : \Gamma^I /
\Gamma^I - \{A\} \Downarrow A^\bot}{}
$$
\vspace{-2.5mm}
$$
\infer[\text{[$D_l$]}]{\vdash
\mathcal{K}^I / \mathcal{K}^O: \Gamma^I / \Gamma^O \Uparrow \cdot}{\vdash
\mathcal{K}^I - \{P\} / \mathcal{K}^O : \Gamma^I / \Gamma^O \Downarrow P}
$$
\vspace{-2.5mm}
$$
\infer[\text{[$D_1$]}]{\vdash \mathcal{K}^I / \mathcal{K}^O : \Gamma^I, P /
\Gamma^O \Uparrow \cdot}{\vdash \mathcal{K}^I / \mathcal{K}^O  : \Gamma^I /
\Gamma^O \Downarrow P}
\qquad
\infer[\text{[$R\Downarrow$]}]{\vdash \mathcal{K}^I / \mathcal{K}^O : \Gamma^I /
\Gamma^O \Downarrow N}{\vdash \mathcal{K}^I / \mathcal{K}^O : \Gamma^I /
\Gamma^O \Uparrow N}
\qquad
\infer[\text{[$R\Uparrow$]}]{\vdash \mathcal{K}^I / \mathcal{K}^O : \Gamma^I /
\Gamma^O \Uparrow L, S}{\vdash \mathcal{K}^I / \mathcal{K}^O : \Gamma^I, S /
\Gamma^O \Uparrow L}
$$
}
\caption{Focused linear logic system with subexponentials with signature
$\langle I, \preceq, \mathcal{W}, \mathcal{C} \rangle$. We assume that: $\mathcal{C} \subseteq
\mathcal{W}$, $\mathcal{K}$ is an indexed context, $L$ is a list of formulas, $\Gamma$ is a multi-set of
formulas and positive literals, $A^\bot$ is a positive polarity literal,
$P$ is a non-negative literal, $S$ is a positive literal or formula and $N$
is a negative formula.}
\label{figure:sellf}
\vspace{-5mm}
\end{figure}

\begin{figure}[t]
% \vspace{-4mm}
{\small
\[
\begin{array}{l@{\qquad}l}
\bullet~(\mathcal{K}_1 \otimes \mathcal{K}_2) [i] = \left\{
\begin{array}{ll}
 \mathcal{K}_1[i] \cup \mathcal{K}_2[i] & \hbox{ if }
i \notin \mathcal{C}\\
 \mathcal{K}_1[i]  & \hbox{ if } i \in \mathcal{C} \cap \mathcal{W}
\end{array}
\right.
& 
\bullet~\mathcal{K}[\mathcal{S}] =
\bigcup\{\mathcal{K}[i]\;|\;i\in \mathcal{S}\}\\[15pt]
\bullet~(\mathcal{K} +_l F) [i] = \left\{
\begin{array}{ll}
 \mathcal{K}[i] \cup \{F\} & \hbox{ if } i = l\\
 \mathcal{K}[i]  & \hbox{ otherwise }
\end{array}
\right.
&
\bullet~ \mathcal{K} \leq_i[l] = \left\{
\begin{array}{ll}
 \mathcal{K}[l] & \hbox{ if } i \preceq l\\
 \emptyset & \hbox{ if } i \npreceq l 
\end{array}
\right.
\end{array}
\]
\[
\bullet~ (\mathcal{K}_1 \star \mathcal{K}_2)\mid_\mathcal{S}
\textrm{ is true if and only if }(\mathcal{K}_1[j]
\star \mathcal{K}_2[j]) \textrm{ for all $j \in \mathcal{S}$.}
\]
}
\caption{Specification of operations on contexts. Here, 
$i \in I$, $\mathcal{S} \subseteq I$, and the 
binary connective $\star \in \{=, \subset, \subseteq\}$.}
\label{Fig:Contexts}
\vspace{-5mm}
\end{figure}



\section*{Future work}

\begin{itemize}
  \item Part $(a)$ of the diagram is done completely by hand. Currently Vivek
  and Ramyaa are developing an \textbf{algorithm to automatize the specification
  process}.

  \item A nice feature would be to \textbf{print a latex code of the system specified in
  SELLF} (would be nice to validate the specification). This can be done using
  the macro-rules of each specification formula and translating them to the object
  logic rule. The generation of macro-rules was implemented at some point, and
  the system actually printed this macro-rule in latex. I would have to check
  again if this code still works and to translate the derivation into a single
  rule.
  
  \item Considering that the specification has a high-level of adequacy, the
  \textbf{proof search in SELLF can be used to do proof search on the object
  logic}. This
  is relatively straightforward to implement and I want to do this for Anna's
  system for paraconsistent logic. We need to uncomment the querying part and
  debug it.

  \item Some time ago, I and Vivek had a project that \textbf{checked automatically
  whether rules of the object logic permute} by making an exhaustive search and
  using answer set programming. The code still exists but there are some bugs
  that we noticed afterwards. This needs to be fixed, but I am afraid that it
  will involve changing the data structure.

  \item Given that there is a way of checking automatically which rules permute
  over which, we could build a permutation graph and try to \textbf{deduce a focused
  proof system for the object logic} from it. This is just an idea and the
  theory behind it should be developed first.

  \item Given that our system proves a lot of things, we think about having
  \textbf{proof objects}. But this is a long term goal, and something that would
  fit Dale Miller's ProofCert project.
\end{itemize}

\end{document}
