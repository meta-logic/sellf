\documentclass{llncs}
\usepackage{latexsym}
\usepackage{amssymb}
\usepackage{amsmath}
%\usepackage{amsthm}
\usepackage{stmaryrd}
\usepackage{proof}
\usepackage{comment}
\usepackage{mathbbol}

\DeclareMathAlphabet{\mathsl}{OT1}{cmr}{m}{sl}
\newcommand\tsl[1]{\hbox{\ensuremath{\mathsl{#1}}}}

\newcommand{\sellf}{\hbox{SELLF}}
\newcommand\zero{0}
\newcommand\one{1}
\newcommand\bottom{\perp}
\newcommand\bang{\mathop{!}}
\newcommand\quest{\mathord{?}}
\newcommand\limp{\mathbin{-\hspace{-0.70mm}\circ}}
\newcommand\tensor\otimes
\newcommand\bla{\mathrel{\mbox{$\circ\!-$}}}
\newcommand\lftall{\forall_{l}}
\newcommand\lexists{\exists_{l}}
\newcommand\lolli{\multimap}
\newcommand\plus{\oplus}
\newcommand\lpar{\mathrel{\bindnasrepma}}
\newcommand\bigpar{\mathrel{\bindnasrepma}}
\newcommand{\with}{\mathbin{\&}}

\newcommand{\nbang}[1]{\hbox{$\bang^\mathsl{#1}$}}
\newcommand{\nquest}[1]{\hbox{$\quest^\mathsl{#1}$}}

\newcommand{\unb}{\tsl{unb}}

\newcommand{\fail}{\hbox{\tsl{fail}}}
\newcommand{\mctx}[2]{\hbox{\tsl{mctx}(\ensuremath{#1},\ensuremath{#2})}}
\newcommand{\elin}[2]{\hbox{\tsl{elin}(\ensuremath{#1},\ensuremath{#2})}}
\newcommand{\emp}[1]{\hbox{\tsl{emp}(\ensuremath{#1})}}
\newcommand{\eqf}[2]{\hbox{\tsl{eqf}(\ensuremath{#1},\ensuremath{#2})}}
\newcommand{\addform}[3]{\hbox{\tsl{addform}(\ensuremath{#1},\ensuremath{#2},\ensuremath{#3})}}
\newcommand{\eqctx}[2]{\hbox{$
\tsl{eqctx}$(\ensuremath{#1},\ensuremath{#2})}}
\newcommand{\union}
[3]{\hbox{\tsl{union}(\ensuremath{#1},\ensuremath{#2},\ensuremath{#3})}}
\newcommand{\In}
[2]{\hbox{\tsl{in}(\ensuremath{#1},\ensuremath{#2})}}
\newcommand{\form}
[1]{\hbox{\tsl{form}(\ensuremath{#1})}}
\newcommand{\Equ}
[2]{\hbox{\tsl{Equ}(\ensuremath{#1},\ensuremath{#2})}}


\newcommand{\mr}[3]{\tsl{mr}(#1,#2,#3)}
\newcommand{\der}[7]{\tsl{mr^2}(#1,#2,#3,#4,#5,#6,#7)}

\newcommand{\Oscr}{\mathcal{O}}
\newcommand{\Cscr}{\mathcal{C}}
\newcommand{\Lscr}{\mathcal{L}}
\newcommand{\Sscr}{\mathcal{S}}
\newcommand{\Tscr}{\mathcal{T}}
\newcommand{\Wscr}{\mathcal{W}}

% For generic contexts
\newcommand{\genK}{\mathcal{K}_{g}}
\newcommand{\genG}[2]{\Gamma_{#1}^{#2}}

\newcommand{\eg}{{\em e.g.}}
\newcommand{\ie}{{\em i.e.}}
\newcommand{\etal}{\emph{et al.}}

\newcommand{\alg}[1]{\mathbb{\Lbrack} #1 \mathbb{\Rbrack}}

% Evironments for comments
\newcommand{\Endmark}{$\diamond$}
\newenvironment{giselle}{\begin{trivlist}\item[]{\bf Giselle:\ }}%
                       {{\hfill \Endmark}\end{trivlist}}

\newenvironment{vivek}{\begin{trivlist}\item[]{\bf Vivek:\ }}%
                       {{\hfill \Endmark}\end{trivlist}}

%opening
\title{Automating the Proof of Permutation Lemmas}
\author{Vivek Nigam \texttt{vivek.nigam at ifi.lmu.de}\\
  Giselle Reis \texttt{giselle at logic.at}}

\institute{Ludwig-Maximilians-Universit{\"a}t, Munich, Germany\\
Technische Universit{\"a}t Wien, Vienna, Austria}


\begin{document}

\maketitle

\begin{abstract}

\end{abstract}

\section{Introduction}

Proof transformation is a powerful technique used in the proof of many
foundational results about proof systems. For instance, one demonstrates
the admissibility of the cut-rule~\cite{gentzen35} by showing
how to transform a proof with cuts into a proof without cuts.
Similarly, in order to show the completeness of a proof search strategy,
such as uniform provability~\cite{miller91apal}  and the focusing
discipline~\cite{andreoli92jlc}, one demonstrates how to transform an
arbitrary (cut-free) proof into another (cut-free) proof that follows the
given strategy, such as an uniform proof or a focused proof. 

However, it is often a tedious task to verify whether a proof
transformation is valid, specially when there is a great number of cases to
consider. For example, in the proof of completeness of the focusing
discipline, one needs to show that some rules permute over other
rules~\cite{miller07cslb}.
These results are called permutation lemmas and they involve
a number of proof transformations. The following transformation is one of
the many cases that need to be considered: A linear logic proof ending with
the following derivation, where $\tensor_R$ is applied last,
{\small
\[
\infer[\tensor_R]{\Gamma, \Gamma' \vdash \Delta, \Delta', A\tensor B,
C\with
D}
{
{\Gamma \vdash \Delta, A}
&
\infer[\with_R]{\Gamma' \vdash \Delta', B, C\with D}
{
{\Gamma' \vdash \Delta', B, C}
&
{\Gamma' \vdash \Delta', B, D}
}
}
\]
}
can be transformed into another linear logic proof ending with the
following
derivation, where the $\with_R$ is applied last:
{\small
\[
\infer[\with_R]{\Gamma, \Gamma' \vdash \Delta, \Delta', A\tensor B, C\with
D}
{
\infer[\tensor_R]{\Gamma, \Gamma' \vdash \Delta, \Delta', A\tensor B, C}
{
{\Gamma \vdash \Delta, A}
&
{\Gamma' \vdash \Delta', B, C}
}
&
\infer[\tensor_R]{\Gamma, \Gamma' \vdash \Delta, \Delta', A\tensor B, D}
{
{\Gamma \vdash \Delta, A}
&
{ \Gamma' \vdash  \Delta', B, D}
}
}
\]
}
The proof transformation above is one of the cases required in showing
that any instance of a $\tensor_R$ rule can permute over any instance
of a $\with_R$. In
particular, to check the correctness of such a transformation, one needs to
check that (1) all rules are correctly applied and that (2) the premises of
the latter derivation can be proved using the proofs introducing the
premises of the former derivation. For instance, in the case above, the
proof introducing the sequent $\Gamma \vdash \Delta, A$ in the former
derivation can be used twice in the latter derivation. 

Although one can check by hand the validity of such proof transformations,
this procedure is prone to human error as one can easily miss a case or
another. A much better approach is to automate this process. However, it
would be easy to build a specific tool that checks the validity
proof transformations for a specific proof system for a given
logic, such as linear logic. Our goal here is a bit more ambitious, as we
propose a general method that can be used for checking proof
transformations for a number of proof systems for different logics, such as
proof systems for intuitionistic, classical, and linear logics. 
Up to our knowledge there is no such a tool yet available.

% \newcommand\rules[1]{\tsl{Rules}{(\ensuremath{#1})}}
% \newcommand\provif[2]{\tsl{Provif}{(\ensuremath{#1}, \ensuremath{#2})}}

We specify proof systems in the linear logic framework with
subexponentials proposed by Nigam \etal\ in \cite{nigam11lsfa}. It was
shown in~\cite{nigam10jar,nigam11lsfa} that a wide range of proof systems,
including natural deduction systems, can be encoded in this
framework. 
Then given a specification of a proof system, this paper shows
how one can check the validity of some proof transformations in the encoded
system by constructing two types of propositional
theories.\footnote{Our theories are also
called answer-set programs~\cite{gelfond90iclp} in the non-monotonic logic
programming community.} The first
theory, $\Tscr$, is constructed for a given inference rule,
$r$, and specifies the set of its valid instances. 
We show that $\Tscr$ is sound and complete in the sense
that its set of \emph{minimal models} corresponds 
exactly to the set of all possible valid instances of the given inference
rule $r$. On the other hand, for two given sequents, $\Sscr_1$ and
$\Sscr_2$, the second theory specifies whether
the sequent $\Sscr_2$ is provable, when assuming that $\Sscr_1$ is also
provable. Although the
latter theory is not complete, but only sound, we use facts about our 
linear logic framework to increase its power. For example,
we are able to show that $\Sscr_2$ is provable,
assuming that $\Sscr_1$ is provable, when the only difference between
$\Sscr_2$ and $\Sscr_1$ is that $\Sscr_2$ has an additional formula that
can be weakened. In the framework proposed in \cite{lutovac.unp}, such
cases seems to require much more machinery.

The main advantage of using propositional theories is that they enable the
use of powerful off-the-shelf
provers~\cite{niemela97lpmnr,leone06tcl}. We
implemented a tool that takes the specification of a proof system and
checks automatically which inference rules of the object-system permute
over another rule. Whenever the tool can find a valid permutation
it outputs the corresponding proof transformation,
and whenever it cannot
show that a rule permutes over another, it outputs the cases that it failed
to find a valid permutation. 
We used this tool to show a number of proof
transformations. For instance, our tool checks all cases of the key
permutation lemmas
needed for showing the completeness of the focusing
discipline~\cite{andreoli92jlc} and uniform proofs~\cite{miller91apal}.
%as well
%as all the general permutation cases
%identified in \cite{lutovac.unp}.

We now summarize our main contributions. Sections~\ref{sec:sellf} and
\ref{sec:encoding_PF}
review the linear logic framework with subexponentials used to encode
proof systems. Sections~\ref{sec:reason} and \ref{sec:provif} detail the
construction the propositional theories described above. We also
prove the
soundness and completeness of the theory that specifies the validity of 
an inference rule, and prove the soundness of
theory specifying the provability relation among sequents. In
Section~\ref{sec:examples}, we report on the use
of existing propositional provers to construct our tool that checks
automatically whether a rule permutes over another. We also summarize our
experimental results on the examples described above.
Finally in Sections~\ref{sec:related} and \ref{sec:conc} we conclude by
discussing related and future work.

% Proof transformation is an important proof theoretic technique that has
% been used for showing a number of foundational results about proof
% systems. For instance, one normally relies on proof transformation to
% show the admissibility of the cut-rule. Similarly,
% proof transformation has also been used to prove the completeness of
% proof
% search strategies, such as uniform provability and the focusing
% discipline. However, in order to check the validity of some proof
% transformation, such as when one inference rule permutes over
% another, one needs to check a number of tedious cases involving a
% number of symbols. Therefore, checking the correctness of proof
% transformations is prone to human error. This paper offers the means to
% automatize this process. In particular, we specify proof systems in the
% logical framework based on linear logic with subexponentials proposed
% previously by the authors. Then, given an specification, we construct a
% classical propositional theory, such that the validity of an instance of
% a
% specified inference rule is reduced to the satisfiability of this logical
% theory. We show that this encoding is sound and complete in the sense
% that the set of minimal models of the propositional theory corresponds 
% exactly to the set of all
% possible valid instances of the given inference rules. Moreover, we also
% provide a theory for checking the provability of a sequent when assuming
% the provability of another sequent an operation needed in many proof
% transformations.
% Finally, we constructed a tool that
% can check automatically which inference rules of a given theory permute
% over each other. Among our examples, we used this tool to prove the key
% permutation lemmas used in the completeness proofs of uniform
% provability of intuitionistic logic and of the focusing discipline of
% linear logic. 
% 
% 
% We show that this encoding is sound and complete, in the sense
% that the set of minimal models of the theory corresponds to the set of
% all
% possible valid instances of the given inference rule.


\begin{comment}
- Logical Frameworks have been proposed and used to specify proof systems
cite previous work of Elaine, Dale, Vivek, and Pfenning. 

- Advantages of using logical frameworks. 

- Proving cut-elimination. 
- Variable binding is precise.
- Connecting different proof systems in a common theory. 

- This paper continues this line of research. In particular, we show how
to derive in an automated fashion, permutation lemmas. Permutation lemmas 
are of great importance for proof search: focusing theorems. Apply
invertible rules first. Moreover, since they involve many cases, they are
prone to human error. 

Main contributions:

- From a logical specification encoding a proof systems, we extract a set 
of set of constraints. Each set of constraints specifies an inference rule 
of the encoded object logic. 

- We show that our algorithm is sound and complete. 

- We show how to relate the provability of two encoded inference rules. 

- We implemented the system and used to automatically enumerate all the
cases for the permutation lemmas of linear logic. 

- Can we show automatically the permutation lemmas of Prawitz's
normalization theorem for Natural Deduction?
\end{comment}



\section{Linear Logic}

Focusing;
Subexponentials.

\section{Encoding Proof Systems}

Linear logic with subexponentials as a logical framework for proof systems.

\section{Macro-rules}

\begin{giselle}
I noticed that in Elaine and Dale's paper, they refer to this as ``big-step
inference rules''. You suggested that we changed this name Vivek, what do you
think about it?
\end{giselle}

Given a specification of a proof system in linear logic with subexponentials, it
is possible to build ``macro-rules'' for each formula that specifies a rule of
the system. A \textbf{macro-rule} of a formula is a partial derivation in
\sellf\ focusing on this formula. This derivation stops when the only possible
inference rules to be applied are either the initial one (corresponds to the end
of the proof search) or one of the decide rules (corresponds to the lost of
focusing of the formula).

\paragraph{Example} Suppose that the rule $\diamondsuit$ of the system is specified by
the formula: $A \oplus (B \otimes C)$, where $A$ and $B$ are negative polarity
atoms and $C$ is a positive polarity atom. The two
possible derivations for this formula are depicted in
Figure~\ref{fig:derivations}\footnote{We make it explicit here that the
derivation starts from the application of a decide rule because we will need
this when generating derivations of more than one macro-rules. (Put reference)}.

\begin{figure}
$$
\infer[D]{\mathcal{K} : \Gamma \Uparrow \cdot}
{
\infer[\oplus_1]{\mathcal{K} : \Gamma \Downarrow A \oplus (B \otimes
C)}{
  \infer[R\Downarrow]{\mathcal{K} : \Gamma \Downarrow A}
  {
    \infer[R\Uparrow]{\mathcal{K} : \Gamma \Uparrow A}
    {\mathcal{K} : \Gamma, A \Uparrow \cdot}
  }
}
}
\;\;\;\;
\infer[D]{\mathcal{K}_1 \otimes \mathcal{K}_2 : \Gamma, \Delta \Uparrow \cdot}
{
\infer[\oplus_2]{\mathcal{K}_1 \otimes \mathcal{K}_2 : \Gamma, \Delta \Downarrow A \oplus (B \otimes C)}
{
  \infer[\otimes]{\mathcal{K}_1 \otimes \mathcal{K}_2 : \Gamma, \Delta \Downarrow B \otimes C}
  {
    \infer[R\Downarrow]{\mathcal{K}_1 : \Gamma \Downarrow B}
      {
        \infer[R\Uparrow]{\mathcal{K}_1 : \Gamma \Uparrow B}{\mathcal{K}_1 :
        \Gamma, B \Uparrow \cdot}
      } \;\;\;\;
    \infer[I]{\mathcal{K}_2 : \Delta \Downarrow C}{}
  }
}
}
$$
\caption{Two possible derivations for the formula $A \oplus (B \otimes C)$.}
\label{fig:derivations}
\end{figure}

% TODO: colocar o end sequent com o pontinho
\begin{figure}
$$
\infer=[]{\mathcal{K} : \Gamma \Downarrow A \oplus (B \otimes C)}{\mathcal{K} :
\Gamma, A \Uparrow \cdot}
\;\;\;\;
\infer=[]{\mathcal{K}_1 \otimes \mathcal{K}_2 : \Gamma, \Delta \Downarrow A \oplus (B \otimes C)}
{
  \mathcal{K}_1 : \Gamma, B \Uparrow \cdot \;\;\;\;
  \infer[]{\mathcal{K}_2 : \Delta \Downarrow C}{}
}
$$
\caption{The two corresponding macro-rules for the derivations of
Figure~\ref{fig:derivations}.}
\label{fig:macro}
\end{figure}


The macro-rules for each of the derivations consist only of its sequent that
focused on the formula and
the leaves, and they are shown in Figure~\ref{fig:macro}. They can be seen as a 
``macro''\footnote{``Macro'' in this context has
the same meaning as when it is used in programming languages.} for the whole 
derivation, therefore the name.

Since the specifications in \sellf have the highest possible level of adequacy, the
application of $\diamondsuit$ on the object logic is equivalent to the derivation of
the formula $A \oplus (B \otimes C)$, which is represented by the two macro-rules
of Figure~\ref{fig:macro}. Thus we can use the macro rules generated from a
specification to do proof search on the specified system, since each of them
would correspond to a sequent calculus inference rule. 
% TODO: GR: maybe elaborate this proof search part, or not cite at all.

In order to build a macro rule for a formula, we must consider that we don't
have any information about the state of the rest of the sequent. Namely, we do
not know what is the content of the context, only how many are there (each
context corresponds to a subexponential of the specification). During the
derivation of the formula, some operations over the context might be applied
(such as the context splitting on the second derivation of our example). To
make sure that all the ``micro'' rules applied are valid and that the leaves'
context reflect these operations we use
\textbf{constraints} (explained in Section~\ref{sec:constraints}) and
\textbf{generic contexts} (Section~\ref{sec:generic_contexts}).

\begin{giselle}
Explain here or later the reason why specifications of sequent calculus systems
will have its formulas decomposed until atoms if we use this terminating
criteria.
\end{giselle}

As we said before, the macro-rules in our setting will be derivations until the
only applicable inference rules are the initial or decide. In the specification
of sequent calculus sytems, this means that the formula is completely decomposed
until atomic level (because they are \textit{bipoles}). 
Each one of these finishing states will generate some
conditions for the proof to continue from there or terminate.
These are some of the constraints added to the macro-rule (in addition to the
ones generated by the application of the ``micro'' rules).



\subsection{Representing Sequents: generic contexts}
\label{sec:generic_contexts}

In order to construct the macro-rules, we must find a good way to represent the
sequents. During proof search, a sequent is composed by the formula (or list of formulas)
focused on and a function from the subexponential indices to a set of formulas
(and a separated set for the formulas in $\Gamma$).

Now suppose that we are deriving the second macro rule of
Figure~\ref{fig:macro}. The macro rule is:

$$
\infer=[]{\mathcal{K}_1 \otimes \mathcal{K}_2 : \Gamma, \Delta \Downarrow A \oplus (B \otimes C)}
{
  \mathcal{K}_1 : \Gamma, B \Uparrow \cdot \;\;\;\;
  \infer[]{\mathcal{K}_2 : \Delta \Downarrow C}{}
}
$$

Of course, the formulas in $\Gamma$ and $\Delta$ are represented by only one
list before the application of the $\otimes$ rule. What is done in proof search
is the lazy split of the original list. This means that if the branch on the
left is resolved first, all formulas are ``carried'' to this branch and after
finishing (if it finishes successufully), the formulas that were left are
``moved'' to the right branch and must be used there. The same approach is used
with the linear subexponentials of $\mathcal{K}$.

The problem of using this solution in deriving a macro-rule is that we do not
really know what is inside the set $\Gamma \cup \Delta$, and the derivation of
the left branch does not even terminate (it could be the case that neither of
them terminates). Nevertheless, we still need
the information that this set was splitted and that some formulas can be used in
one branch and others on the other, but not in both. In particular, we must
ensure that a formula $C^{\bot}$, if it exists, will go to the right branch so
that the initial rule can be applied. To solve this problem we
use \textit{generic contexts}.

Generic contexts are ``new'' lists of formulas that are created during the
derivation of a macro-rule and are often included on the constraints. On the
example above, let $\Gamma_0 = \Gamma \cup \Delta$. On the application of rule
$\otimes$, the new lists $\Gamma$ and $\Delta$ will be created and a constraint
that $\Gamma \cup \Delta$ is equal to $\Gamma_0$ will be added to the
macro-rule. This solution is also used for the linear subexponentials of
$\mathcal{K}$.

\begin{giselle}
On the next paragraph, I am suggesting a notation for the generic context, but I
am not sure if it's the one used on all the examples. We should check this and
make the document consistent.
\end{giselle}

Initially, there will be one list for each subexponential and another list for
the purely linear formulas ($\Gamma$). Throughout this document, we will
represent these lists by $\genG{s}{i}$, where $i$ is an index used to keep track
of the created sets and $s$ is the subexponential name or it
is not present if it is actually $\Gamma$ (the set of formulas that are not marked with
$?$). The set $\mathcal{K}$ will be represented as $\genK$ when we are refering
that it contains generic contexts instead of sets of formulas.
%Everytime a list is splitted or modified, we will mark it's parts or the
%modified list with a superscript, e.g. $\Gamma_s^1$ and $\Gamma_s^2$.

We call the generic contexts of a sequent a \textit{side-formula context} (because it
contains only side-formulas of the sequent) and this is represented as a
function from the set of subexponential indices to a natural number.

% GR: it does not go to a set of formulas. We will only use those on the
% constraints.

A sequent is represented by a side-formula context and a formula (or list of
formulas) focused on.



\begin{comment}
\begin{giselle}
Maybe we don't need to make a distinction between the unbounded context or the
others... since it is declared in the specification just like the other
subexponentials. Do we?
\end{giselle}

We assume that the names of the contexts in the object-logic are specified
as subexponential names and that the meta-logic has an unbounded
subexponential, referred to as \unb, that may only contain the clauses specifying 
the inference rules of the object-logic.

A side-formula context is represented as a function from the set of
subexponential 
indexes to a set of \sellf\ formulas and generic context names, such as 
$\Gamma$. 
\end{comment}

\subsection{Constraints}
\label{sec:constraints}

We will make use of a set with the following type of constraints to
represent in a generic fashion a macro-rule in \sellf:

\vspace{0.5cm}
\begin{tabular}{|l|p{8cm}|}
\hline
$\fail$ & Denotes that the set of constraints is
unsatisfiable.\\
\hline
$\mctx{F}{\Gamma}$ & Denotes that the formula $F$ is in the
unbounded context $\Gamma$, that is, $F \in \Gamma$.\\
\hline
$\elin{F}{\Gamma}$ & Denotes that the formula $F$ is the
only element in the linear context $\Gamma$, that is, $\Gamma = \{F\}$.\\
\hline
$\emp{\Gamma}$ & Denotes that the context $\Gamma$ is empty,
that is, $\Gamma = \emptyset$.\\
\hline
$\eqf{F}{G}$ & Denotes that the formulas $F$ and $G$ are the
same \emph{occurrence} of the same formula.\\
\hline
$\eqctx{\Gamma}{\Gamma'}$ & Denotes that the contexts $\Gamma$ and
$\Gamma'$ contain the same elements.\\
\hline
$\union{\Gamma'} {\Gamma''} {\Gamma}$ & Denotes that the bounded context
$\Gamma = \Gamma' \cup \Gamma''$.\\
\hline
$\addform{F}{\Gamma}{\Gamma'}$ & Denotes that the unbounded context $\Gamma'$ is the result of adding the formula $F$ to $\Gamma$.\\
\hline
\end{tabular}
\vspace{0.5cm}

These constraints are defined by the following clauses:

\begin{itemize}

\item $\union{\Gamma'}{\Gamma''}{\Gamma}$
\[
\begin{array}{l}
\In{X}{\Gamma} \subset \In{X}{\Gamma'}, \union{\Gamma'}{\Gamma''}{\Gamma},
\form{X}. \\
\In{X}{\Gamma} \subset \In{X}{\Gamma''}, \union{\Gamma'}{\Gamma''}{\Gamma},
\form{X}. \\
1~ \{\In{X}{\Gamma'}, \In{X}{\Gamma''}\}~ 1  \subset \In{X}{\Gamma},
\union{\Gamma'}{\Gamma''}{\Gamma}, \form{X}.\\
\end{array}
\]
where $1~ \{A, B\}~ 1$ denotes that either the atom $A$ or the atom $B$
is true.

\item $\elin{F}{\Gamma}$
\[
 \begin{array}{l}
  \In{F}{\Gamma} \subset \elin{F}{\Gamma}.\\
  \bot \subset \In{X}{\Gamma}, \tsl{not} \Equ{X}{F}, \form{X},
\elin{F}{\Gamma}.
 \end{array}
\]

\item $\emp{\Gamma}$
\[
 \begin{array}{l}
  \bot \subset \In{X}{\Gamma}, \form{X},\emp{\Gamma}.
 \end{array}
\]

\item $\mctx{F}{\Gamma}$
\[
\begin{array}{l}
  \In{F}{\Gamma} \subset \mctx{F}{\Gamma}.\\
\end{array}
\]

\item $\eqctx{\Gamma}{\Gamma'}$
\[
 \begin{array}{l}
  \In{F}{\Gamma} \subset \In{F}{\Gamma'}, \eqctx{\Gamma}{\Gamma'},
\form{F}.\\
  \In{F}{\Gamma'} \subset \In{F}{\Gamma}, \eqctx{\Gamma}{\Gamma'},
\form{F}.
 \end{array}
\]

\item $\addform{F}{\Gamma}{\Gamma'}$
\[
\begin{array}{l}
\mctx{F}{\Gamma'} \subset \addform{F}{\Gamma}{\Gamma'}, \form{F}.\\
\mctx{F'}{\Gamma'} \subset \addform{F}{\Gamma}{\Gamma'}, \mctx{F'}{\Gamma},
\form{F}, \form{F}.\\
\end{array}
\]

\end{itemize}

Besides the clauses above, the following are added for consistency:

$$
\begin{array}{l}
\emp{\Gamma'} \subset \emp{\Gamma}, \union{\Gamma'}{\Gamma''}{\Gamma}. \\
\emp{\Gamma''} \subset \emp{\Gamma}, \union{\Gamma'}{\Gamma''}{\Gamma}. \\
1 \{ \elin{F}{\Gamma'}, \elin{F}{\Gamma''} \} 1 \subset \elin{F}{\Gamma},
\union{\Gamma'}{\Gamma''}{\Gamma}. \\
1 \{ \emp{\Gamma'}, \emp{\Gamma''} \} 1 \subset \elin{F}{\Gamma},
\union{\Gamma'}{\Gamma''}{\Gamma}. \\
\end{array}
$$

From the constraint specifying the union of two generic contexts and 
the constraint specifying that two contexts have the same elements, 
we can construct the more general constraint:
\[
 \eqctx{\{\Gamma_1, \ldots, \Gamma_n\}}{\{\Gamma_1', \ldots, \Gamma_m'\}}
\]
denoting that $\Gamma_1 \cup \cdots \cup \Gamma_n = \Gamma_1' \cup \cdots 
\Gamma_m'$. We first specify that $\Delta = \Gamma_1 \cup \cdots \cup
\Gamma_n$ and $\Delta' = \Gamma_1' \cup \cdots \cup
\Gamma_n'$, where $\Delta$ and $\Delta'$ are fresh generic contexts.
These constraints are specified inductively by introducing new auxiliary 
generic contexts. The base case is when there is a single generic context, 
$\Gamma$, in which case we return $\Gamma$. Otherwise pick two generic
contexts $\Gamma_i$ and $\Gamma_j$ and create a new generic context
$\Delta_{i,j}$ and specify $\Delta_{i,j} = \Gamma_i \cup \Gamma_j$ using 
the constraint \union{\Gamma_i}{\Gamma_j}{\Delta_{i,j}}. Then we replace 
both $\Gamma_i$ and $\Gamma_i$ by a single occurrence of $\Delta_{i,j}$.
We repeat this procedure with the resulting set of generic contexts. 
We apply this operation to both  $\{\Gamma_1, \ldots, \Gamma_n\}$ and
$\{\Gamma_1, \ldots, \Gamma_n\}$ obtaining two generic contexts $\Delta$
and $\Delta'$ and two sets of union constraints $\Tscr$ and $\Tscr'$. Then
we add the constraint $\eqctx{\Delta}{\Delta'}$ to the set $\Tscr \cup
\Tscr'$. 

% LEMMA: mostrar que se uma formula esta no delta final de um lado ela tb esta
% no delta final do outro lado. Prova por inducao no tamanho do set.



\subsection{Macro-rules}
\label{sec:macro_def}

\begin{giselle}
TODO: make the notation of macro rule consistent over the document.
\end{giselle}

After introducing the concepts of side-context, generic contexts and
constraints, we can define a macro-rule in our setting.

A \textit{macro-rule} is a function that takes a formula $F$, a side-context
$\Sscr$ and a set of sets of constraints $\Tscr_0$ and returns a tuple:

$$\mr{F}{\Sscr}{\Tscr_0} = \langle \Cscr, \Oscr, \Tscr \rangle$$

where $\Cscr$ is a set of closed leaves (sequents), $\Oscr$ is a set of open
leaves (sequents) and $\Tscr$ is a \underline{set of sets} of constraints. We
will represent \underline{sets} of constraints by the letter $T$ possibly
subscripted.

%% G: Introducing the notion of instance of macro-rule in this paragraph to
% avoid confusion. A formula has two (or more) different macro-rules when
% there's a non-deterministic choice involved (such as the additive or
% operator), but it has two instances when it can be satisfied by different
% constraints at the top level (such as an atom being on the classical or linear
% context).
Intuitively, each satisfiable set of constraints $T_i$ in $\Tscr$ specifies one possible
\underline{instance} of the macro-rule that can introduce the formula $F$. Instances of a
macro-rule consist of the same set of open and closed leaves but a different
configuration of the context. For example, if the macro-rule ends with the
sequent: $\genK : \Gamma^i \Downarrow A$, it might be the case that
$A^{\bot}$ is in $\Gamma^i$ or in one of the subexponentials. Each of these cases,
if it can occur given the previous constraints, consist of an instance of this
macro-rule. Therefore, the set $\Tscr$ will contain one set for each one of
these possibilities. On the other hand, if there are more than one 
macro-rule that can introduce a
formula, as it was the case of our example with $A \oplus (B \otimes C)$, these
are represented by two different tuples, say $mr_1(F, \Sscr, \Tscr_0)$ and
$mr_2(F, \Sscr, \Tscr_0)$, since the
sets of open and closed leaves are different.

Given a formula $F$, a side-context $\Sscr$ and a set of sets of constraints
$\Tscr_0$, we construct its macro-rules according to the rules described below.
The derivation ends when the leaves are either closed (and are put in $\Cscr$) or
open but of the form $\genK : \Gamma^i \Uparrow \cdot$ (and are put in $\Oscr$).

If we don't have any previous information about $\Sscr$, we build it by
assigning one generic context for each subexponential and one for $\Gamma$ and
giving the index 0 for all of them. If we don't have any initial constraints,
then $\Tscr_0$ is set to $\emptyset$.

The initial sequent for the derivation of a macro-rule are the generic contexts
for subexponentials, represented by $\genK$, and the one for $\Gamma$,
represented by $\Gamma^i$.

We can divide the cases into negative rules and positive
rules. The order of application of these rules will follow the focusing
discipline, i.e., the negative rules are applied whenever possible on the open
sequents, just like it would be done in proof search.

We will denote the rules of this algorithm by $\alg{\circ}$, where $\circ$ is
the connective or name used in \sellf.

\paragraph{$\alg{I}$: $\mathbf{\vdash \genK : \Gamma^i \Downarrow A}$\\} 
$A$ is a positive atom in a
synchronous phase. In that case, there is only one possibility, which is to
close the sequent. For that, one of two things must occur:
\begin{itemize}
  \item $A^{\bot}$ is in $\Gamma^i$ and every context that cannot suffer
  weakening in $\genK$ is empty. This is represented by the following
  constraint:

  $$T_u = \{\elin{A^{\bot}}{\Gamma^i}\} \cup \{ \emp{\Gamma_s} \mid \forall s
  \notin \Wscr\} $$

  where $\Wscr$ is the set of subexponential indices that can suffer weakening. 

  \item $A^{\bot}$ is in one of the subexponentials, say $s$. Again, we have two cases:
  either $s$ can suffer weakening (in which case the formula can co-exist with
  others) or $s$ cannot suffer weakening (in which case $A^{\bot}$ must be the
  only formula in $s$). Each of the possible subexponentials will generate one different
  set of constraints. For each subexponential considered that cannot suffer weakening, the
  constraints are the following:

  $$T_l = \{\elin{A^{\bot}}{\Gamma_l} \} \cup \{ \emp{\Gamma_s} \mid \forall s
  \neq l \wedge s \notin \Wscr \} \cup \{ \emp{\Gamma^i} \}$$

  And for each subexponential considered that can suffer weakening, the
  constraints are the following:  

  $$T_w = \{\mctx{A^{\bot}}{\Gamma_w} \} \cup \{ \emp{\Gamma_s} \mid \forall s
  \notin \Wscr \} \cup \{ \emp{\Gamma^i} \}$$

  As before, $\Wscr$ represents the set of subexponential indices that can
  suffer weakening.
\end{itemize}

Assuming that $\Tscr = \{ T_1, ..., T_n \}$ was the set of set of constraints so
far, the new set of constraints for this macro-rule the are the sets resulting
from the union of each $T_i \in \Tscr$ with each $T_u$, $T_l$ and $T_w$.

\paragraph{$\alg{R\Uparrow}$: $\mathbf{\genK : \Gamma^i, A \Uparrow \cdot}$\\} 
In order to come to this configuration, it must be the case that some literal
(possibily negative) or
positive formula $A$ was put in the context by the $R\Uparrow$ rule. As we are
dealing with specifications of sequent calculus systems, we stress that it must
be a literal, and never a positive formula. This happens because the formulas
that specify the inference rules of such systems are always
\textit{bipoles}~\cite{paper da Elaine e Dale? da Agata?}. Roughly, this means
that there are no positive formulas as subformulas of negative ones, so when
negative formulas are being decomposed its subformulas are either other negative
formulas (in this way focus is not lost) or literals.

\begin{giselle}
I am putting this observation above about atomic formulas and bipoles because
that's what we are dealing with at the moment, but there are no restrictions on
the method if $A$ were a formula. In that case, the only thing that would happen
is that the rule would not be specified by only one macro-rule, since it was not
completely decomposed.
\end{giselle}

If the end sequent is of this form, what needs to be done regarding the
constraints is add the formula $A$ to $\Gamma^i$. Let's say that the last number
assigned for $\Gamma$ in the derivation so far is $n$. Two new generic
contexts, $\Gamma^{n+1}$ and $\Gamma^{n+2}$ will be created and 
the following set of contraints should be added to each element of $\Tscr$:

$$T = \{ \elin{A}{\Gamma^{n+1}}, \union{\Gamma^i}{\Gamma^{n+1}}{\Gamma^{n+2}} \}$$

And the new number of $\Gamma$ on this end sequent will be $n+2$.

\paragraph{$\alg{\binampersand}$: $\mathbf{F = A\ \binampersand\ B}$\\}
Therefore, it is the case that $\Sscr = \genK : \Gamma^i \Uparrow L, A\
\binampersand\ B$. Applying the inference rule for $\binampersand$ corresponds to
removing $\Sscr$ from $\Oscr$ and adding the following open sequents to the set:
$\genK : \Gamma^i \Uparrow L, A$ and $\genK : \Gamma^i \Uparrow L, B$. The
generic contexts of $\Sscr$ are copied to each new sequent.

\paragraph{$\alg{\bindnasrepma}$: $\mathbf{F = A\ \bindnasrepma\ B}$\\}
In this case $\Sscr = \genK : \Gamma^i \Uparrow L, A\ \bindnasrepma\ B$, and
the application of rule $[\bindnasrepma]$ corresponds simply to replacing
$\Sscr$ in $\Oscr$ by $\genK : \Gamma^i \Uparrow L, A, B$.

\paragraph{$\alg{\bot}$: $\mathbf{F = \bot}$\\}
As before, the application of rule $[\bot]$ will only involve replacing the
sequent $\Sscr = \genK : \Gamma^i \Uparrow L, \bot$ by $\genK : \Gamma^i
\Uparrow L$.

\begin{comment}
\paragraph{$\mathbf{F = \forall x. A}$\\}
Again, the application of rule $[\forall]$ will correspond with the sustitution
of the sequent $\Sscr = \genK : \Gamma^i \Uparrow L, \forall x. A$ in
$\Oscr$ with $\genK : \Gamma^i \Uparrow L, A\{c/x\}$, where $c$ is a fresh
variable.
\end{comment}

\paragraph{$\alg{\top}$: $\mathbf{F = \top}$\\}
In the case that $\Sscr = \genK : \Gamma^i \Uparrow L, \top$, all that needs
to be done is remove this sequent from $\Oscr$ and add it to $\Cscr$, since it
concludes a derivation and it is therefore a closed sequent.

\paragraph{$\alg{?^l}$: $\mathbf{F = ?^l A}$\\}
This case will involve a change on the constraints. The inference rule for $?^l$
involves the inclusion of formula $A$ in one of the subexponentials, so we have
to take this into consideration for the next sequent. Suppose that the number of
the generic context for $l$ in the macro-rule so far is $i$. We have two possible cases,
either this is a linear subexponential or it's an unbounded one. In both cases
we will need an auxiliary generic context $\Gamma_l^{i+1}$. For the linear case,
the new constraint will be:

$$T_l = \{ \elin{A}{\Gamma_l^{i+1}} \cup
\union{\Gamma_l^{i}}{\Gamma_l^{i+1}}{\Gamma_l^{i+2}} \}$$

For the unbounded case, we use the $addform$ constraint:

$$T_u = \{ \addform{A}{\Gamma_l^{i}}{\Gamma_l^{i+1}} \}$$

In both cases, the new constraint $T_l$ or $T_u$ is added to all sets in
$\Tscr$.

\vspace{0.5cm}

Those were the negative cases. The positive cases follow.

% restrict the terms that can be used to instantiate the existential (do
% not use the terms created by forall rules)

% Do not worry about first order cases by now.
\begin{comment}
\paragraph{$\alg{\exists}$: $\mathbf{F = \exists x. A}$\\}
In this case, $\Sscr = \genK : \Gamma^i \Downarrow \exists x. A$ and the
application of the inference rule $[\exists]$ will replace $\Sscr$ in $\Oscr$ by
$\genK : \Gamma^i \Downarrow A\{t/x\}$, where $t$ is a new variable.
\end{comment}

\paragraph{$\alg{1}$: $\mathbf{F = 1}$\\}
If the focused formula is $1$, we can consider this sequent as a closed one, but
notice that the inference rule has the emptiness of $\Gamma^i$ as a side
condition. Therefore the constraint $\emp{\Gamma^i}$ is added to every set in $\Tscr$.

\paragraph{$\alg{!^l}$: $\mathbf{F = !^l A}$\\}
In this case, $\Sscr = \genK : \Gamma^i \Downarrow !^l A$ and the inference
rule to be applied is $[!^l]$. This rule has the emptiness of some sets as a
side condition, and this will generate some constraints to be added to all sets
in $\Tscr$. First of all, $\Gamma^i$ must be empty, so the following constraint 
must hold: $\emp{\Gamma^i}$. Also,
every subexponential that is less than or not related to $l$ in the parcial
order defined by the subexponential signature must be empty (note that this
subexponential index must not allow contraction). So, for each subexponential
$x$ that satisfies this criteria ($x$ is less than or not related to $l$ and $x$
cannot suffer weakening), the following constraint is generated:
$\emp{\Gamma_x^{i_x}}$ (considering that $i_x$ is the index of the generic
context of subexponential $x$ is $\Sscr$). The idea of this side condition is to
avoid that formulas that cannot suffer weakening are not erased by the operation
$K_{\leq_l}$. This operation will be represented in our setting with the
instantiation of new generic contexts for those that should have their formulas
erased, namely, subexponentials $y$ such that $y$ is less than or not related to
$l$ and $y$ \underline{can} suffer weakening. This is done by incrementing the
index $i_y$ of each $y$ and adding the constraint $\emp{\Gamma_y^{i_y + 1}}$.

Summarizing the operations, for the application of the $[!^l]$ rule we generate
the following constraints:

\begin{itemize}
    \item $T_l = \{ \emp{\Gamma^i} \}$
    \item $T_x = \{ \emp{\Gamma_x^{i_x}} | \forall x. (x \preceq l \vee x \text{ is
    not related to } l) \wedge x \notin \Wscr \}$
    \item $T_y = \{ \emp{\Gamma_y^{i_y + 1}} | \forall y. (y \preceq l \vee y \text{ is
    not related to } l) \wedge y \in \Wscr \}$
\end{itemize}

We also must increment the index $i_y$ of each generic context $\Gamma_y$ as
defined above. The set $T = T_l \cup T_x \cup T_y$ is added to each set in
$\Tscr$. The sequent $\Sscr$ in $\Oscr$ is replaced by $\genK' : \Gamma
\Uparrow A$, where $\genK'$ are the generic contexts modified as
described.

% TODO: GR: modify the explanation of indexes. We need a global counter because of
% the tensor rule... review everything above =(
% Actually, we need a global counter and we also need the local number on each
% sequent... oh god, what a mess... I need to explain this right.

\paragraph{$\alg{\otimes}$: $\mathbf{F = A\ \otimes\ B}$\\}
For the $\otimes$ introduction rule the context must be splitted, so we create a
new pair of generic contexts for each subexponential that cannot suffer
contraction\footnote{If the subexponential can suffer contraction, we can copy
all the formulas to both premisses.}, and add a union constraint to relate the
generic contexts. Suppose that $s$ is a subexponential that cannot suffer
contraction, so it's generic context must be splited. Let $i_{sg}$ be the global
index of $s$ and $i_s$ be the index in $\Sscr$. 
Two new generic contexts $\Gamma_s^{i_{sg} + 1}$ and $\Gamma_s^{i_{sg} +
2}$ are created and the constraint $\union{\Gamma_s^{i_s}}{\Gamma_s^{i_{sg} +
1}}{\Gamma_s^{i_{sg} + 2}}$ is added to each set in $\Tscr$. The new index of
the generic context for $s$ in each of the premisses are now $i_{sg} + 1$ and
$i_{sg} + 2$. In this rule we notice the need to keep a global index counter for
each generic context. The operation described above is executed for every
subexponential that cannot suffer weakening and for the set $\Gamma$, and the sequent $\Sscr =
\genK : \Gamma \Downarrow A\ \otimes\ B$ in $\Oscr$ is substituted by
$\genK' : \Gamma' \Downarrow A$ and $\genK'' : \Gamma'' \Downarrow
B$, where $\genK'$, $\genK''$, $\Gamma'$ and $\Gamma''$ are the
generic contexts with the new indexes as we described.

\paragraph{$\alg{\oplus}$: $\mathbf{F = A\ \oplus\ B}$\\}
The $\oplus$ introduction rule is a particular one. It is responsible for
generating multiple macro-rules for the same formula since it involves a choice
of one of the formulas $A$ or $B$ to continue with. Suppose that this formula is
encoutered in a macro-rule $\langle \Cscr, \Oscr, \Tscr \rangle$, where the
sequent $\Sscr = \genK : \Gamma \Downarrow A\ \oplus\ B$ is in $\Oscr$.
The way to procede in this case is to duplicate the macro-rule and work
separately on the new ones. The new macro-rules are:

$$mr' = \langle \Cscr, \Oscr', \Tscr \rangle$$
$$mr'' = \langle \Cscr, \Oscr'', \Tscr \rangle$$

where $\Oscr'$ is $(\Oscr \backslash \Sscr) \cup \{ \genK : \Gamma
\Downarrow A \}$ and $\Oscr''$ is $(\Oscr \backslash \Sscr) \cup \{ \genK : \Gamma
\Downarrow B \}$. 

\vspace{0.5cm}

Using the previous description of how a macro-rule is build, it is easy to
implement an algorithm from it.




\begin{comment}
Then, the set of macro-rules,
$\mr{F}{\Cscr}{\Oscr}{\Tscr}{\Sscr}$, that can introduce a formula, $F$,
given a root side-formula context $\Sscr$, is represented by a list of
closed leaves
$\Cscr$, a list of open leaves $\Oscr$, and a set of sets of constraints,
$\Tscr$.
\end{comment}

\begin{comment}
Given a formula, $F$, and a root side-formula context, $\Sscr$, we
macro-rules of a formula inductively. Our procedure will have an invariant
that the unbounded contexts will have exactly one generic context. This
does not mean, however, that an unbounded context might not have more than 
a formula. 

There are two base cases: The first base case is when a derivation
finishes with an initial rule. In this case $F = A^\bot$. The set of closed
and open leaves are, respectively,  $\Cscr = \{S\}$ and $\Oscr
= \emptyset$. We construct the set of constraints as follows. Assume that
$u_1, \ldots, u_n$ are the unbounded
subexponentials and $\Lscr = l_1, \ldots, l_m$ are the linear
subexponentials. In order for the initial rule to be applicable $A$ must be
present in some context of the side-formula context $\Sscr$. There are two 
cases, $A$ appears in an unbounded context or $A$ is the only element in 
a linear context. 

If $A$ appears in an unbounded context $\tsl{u}_j$, then 
all the linear contexts must contain only generic contexts, $\Gamma_{l_i}$.
% G: Why generic contexts? Can't we just say that all linear context must be
% empty? They are always generic in some sense...
% VN: Here what I meant is that no formulas can appear in linear contexts. 
% Remember that we have two types of objects that can appear in contexts:
% generic contexts and formulas. 
We construct the constraints specifying that the linear contexts are all
empty as follows: $\Tscr_l = \{\emp{\Gamma_i} \mid i \in \Lscr \}$.
If any linear context contains a formula, then the macro-rule is not valid.
% G: Do we really have to include this fail by ourselves? Won't smodels realize
% it's unsatisfiable if there is a constraint elin(X, \Gamma_i)?
% VN: There are two ways of handling this. One is transforming Gamma, F
% into a new generic context Delta, and then specifying that emp(Delta)
% or we do this by hand by using the fail constraint.
% GR: Do you mean that if we have the following constraints co-existing: elin(X,
% \Gamma_i) and emp(\Gamma_i) it will not fail by itself?
In this case,  $\Tscr_l = \{ \fail \}$, that is, it is macro-rule that 
is not satisfiable.
Moreover, if the context for $\tsl{u}$ is $\Gamma_u, F_1, \ldots, F_{k_u}$,
then there are $k_u+1$ possible macro-rules: either $A \in \Gamma_{u_j}$ or
$A = F_i$, for $1 \leq i \leq k_u$. These possible macro-rules are again
specified by using
constraints:
$\Tscr_{u_j}^{\Gamma_u} = \{\mctx{F}{\Gamma_u}\}$, and
$\Tscr_{u_j}^i = \{\eqf{F}{F_i}\}$ for $1 \leq i \leq k$.

\[\Tscr_u = \bigcup_{1\leq j\leq n} \left\{\{\Tscr_{u_j}^{\Gamma_{u_j}}
\cup \Tscr_l \} \cup \bigcup_{1 \leq i \leq k_{u_j}}\{\Tscr_{u_j}^i \cup
\Tscr_l\}\right\}.\]

If $A$ is in a linear context, say $l_j$, then $A$ must be the only 
formula in that context. Hence, the context of $l_j$ may contain 
a generic context, $\Gamma_{l_j}$, or a context and a single formula
$F_{l_j}$. 
In the former case, $\Gamma_{l_j}$ contains only one formula $A$, which 
is specified by the constraint
$\elin{F}{\Gamma_{l_j}}$, while in the latter case, $\Gamma_{l_j} =
\emptyset$ and $A = F_{l_j}$, which is specified by the constraints 
$\{\emp{\Gamma_{l_j}}, \eqf{A}{F_{l_j}}\}$. Moreover, all the other 
linear context must be empty, which is specified by the set of constraints
$\Tscr_{l_j} = \Tscr_l \setminus \{\emp{\Gamma_{l_j}}\}$. If any
context contains a formula, then the macro-rule is not valid. In this
case,  $\Tscr_{l_j} = \{ \fail \}$. 
Hence, we obtain the following set of set of constraints:

\[\Tscr_l = \bigcup_{1\leq j\leq m} \{\{\elin{F}{\Gamma_{l_j}}\} \cup
\Tscr_{l_j}\} \cup \{\{\emp{\Gamma_{l_j}}, \eqf{A}{F_{l_j}}\} \cup
\Tscr_{l_j}\}.\]
Finally the set of constraints is $\Tscr = \Tscr_u \cup \Tscr_l$. 

% G: Should I stop if I encounter a positive formula??
% I thought we were stopping when we got to atomic principal formulas, since
% this means that the whole clause was decomposed...
The
other base cases result at the end of the 
negative phase, in which case the set of closed and open leaves are,
respectively, $\Cscr = \emptyset$ and $\Oscr = \{S\}$. Moreover, there
are no constraints, that is, $\Tscr = \emptyset$.
\end{comment}

\begin{comment}
For the negative rules, there are no constraints generated nor does
one need to change the root side-formula context.  One
simply appends the list of open and closed leaves, whenever needed, \eg, 
for the $\with$ rule. 
\begin{giselle}
Not entirely true... rule ?l generates a constraint.
\end{giselle}
\end{comment} 

\begin{comment}
For the positive rules, each requires some special 
attention. In particular, for the introduction rule for the existential 
quantifier, one simply instantiates the existentially quantified variable 
with a new variable, so that the resulting formula is uniquely identified. 
The bang introduction rule one changes the root side-formula context of
the premise
so that the side condition of the bang rule is satisfied. In the process
some contexts for unbounded contexts are deleted and some new constraints
of the form $\emp{\Gamma_l}$ might be added for linear contexts
$\Gamma_l$. If a linear context that should be empty contains a formula, 
then the constraint \fail\ is added instead.

The case that one needs to take a bit more care is with the $\tensor$
introduction rule. In particular,  we create a pair of new
generic context for each linear 
subexponential. They are split among the premises of the tensor. On the 
other hand, the contexts for the unbounded contexts are copied to the
premises. We illustrate this case with an example.
Assume that there are only three
subexponentials, one unbounded \tsl{u} and two linear \tsl{l_1} and 
\tsl{l_2}:

% G: Do we need to show the formulas explicitly? Why?
\[
 \infer[\tensor]{\vdash \Gamma_u : \Gamma_{l_1}, \Delta_1 :
\Gamma_{l_2}, \Delta_2: \cdot \Downarrow F_1 \tensor F_2}
{
\vdash \Gamma_u : \Gamma_{l_1}' : \Gamma_{l_2}': \cdot \Downarrow F_1
&
\vdash \Gamma_u : \Gamma_{l_1}'' : \Gamma_{l_2}'': \cdot \Downarrow F_2
}
\]
where $\Delta_i$ are sequences of formulas. 
Notice that the unbounded context $\Gamma_u$ is copied among the premises
and the
new linear contexts $\Gamma_{l_1}', \Gamma_{l_1}'', \Gamma_{l_2}',$ and
$\Gamma_{l_2}''$ are split among the premises. Moreover, that the
formulas in $\Delta_i$ do not appear anymore in the premises. 

For the set of open and closed leaves, we proceed as before and
simply append the set of open and closed leaves obtained using the premises
above as root side-formula contexts. 

For the set of constraints, we proceed in two steps. First, we merge the
constraints obtained by the rule's premises as follows: Assume
$\Tscr_1$ and $\Tscr_2$ are the set of
sets of constraints of the premises of the tensor rule. Then the
constraints of the tensor should include the set:
\[
 \Tscr' = \{T_1 \cup T_2 \mid T_1 \in \Tscr_1 \textrm{ and } T_2 \in
\Tscr_2\}
\]
Intuitively, any macro-rule introducing first premise together with
any macro-rule introducing the second premise is a valid macro-rule for 
the tensor. 

However, we still need to express the relation between the linear 
contexts in the conclusion side-formula context and the new linear contexts
in the premise side-formula contexts. In particular, we want to specify
that $\Gamma_{l_i} \cup \Delta_i = \Gamma_{l_i}' \cup \Gamma_{l_i}''$, 
which could be specified by
constraints of the form
$\eqctx{\{\Gamma_{l_i}, \Delta_i\}}{\{\Gamma_{l_i}', \Gamma_{l_i}'' \}}$.
However, the formulas in $\Delta_i$ are not generic contexts and therefore
this constraint is not well-formed. Therefore, we pre-process 
the set $\Gamma_{l_i} \cup \Delta_i$ so to obtain a set containing only 
generic contexts. This is done by the following operation that introduces
inductively new generic contexts:
if $\Delta_i = \emptyset$ then we are done; otherwise let $F \in
\Delta_i$. Then we introduce two new generic contexts $\Gamma_{l_i}^1$ and 
$\Gamma_{l_i}^F$,
such that $\Gamma_{l_i}^1 = \Gamma_{l_i} \cup \Gamma_{l_i}^F$
specified by the
constraint \union{\Gamma_{l_i}}{\Gamma_{l_i}^F}{\Gamma_{l_i}^1} and that 
$\Gamma_{l_i}^F = \{F\}$, specified by the constraint
\elin{F}{\Gamma_{l_i}^F}. We repeat this operation to the set 
$\Gamma_{l_i}^1 \cup \Delta_i\setminus\{F\}$ that contains less formulas. 
At the end we add all the resulting constraints to all the sets of
constraints in the set $\Tscr'$. 
\end{comment}

% TODO: GR: check this example because somethings changed on the generation of the
% macro-rule.

\paragraph{Example} In order to illustrate the definitions above, consider
extracting the macro-rules for the following 
formula using the algorithm proposed previously: 
\[\exists x. [(A\,x)^\bot \tensor (\nbang{r} \nquest{l} B\,x)].
\]
Moreover, assume that there is only one unbounded context $u$ and two 
linear contexts $l$ and $r$. Assume that the root side-formula context is
$\vdash \Gamma_u : \Gamma_r, F : \Gamma_l$, where all subexponentials have
only 
generic contexts, except for the context of $r$, which contains the formula
$F$. Also assume that $l, r \preceq u$. We proceed by introducing the
existential quantifier, which causes 
$x$ to be replaced by a new value, $t$: 
\[
 (A\,t)^\bot \tensor (\nbang{r} \nquest{l} B\,t).
\]
The root context does not change and no constraints are added. To introduce
the tensor, we first have to replace formulas appearing in the side-formula
context by 
generic context. In particular, we replace $\Gamma_r \cup \{F\}$ by the 
context $\Gamma_r'$ and add the constraints:
\union{\Gamma_r}{\Gamma_F}{\Gamma_r^1} and \elin{F}{\Gamma_F}. Then we
create a pair of new contexts for each linear context: $\Gamma_r',
\Gamma_r'', \Gamma_l',$ and $\Gamma_l''$ and add the constraints 
\union{\Gamma_r'}{\Gamma_r''}{\Gamma_r^1} and
\union{\Gamma_l'}{\Gamma_l''}{\Gamma_l}. The resulting premises are 
the following: $\vdash \Gamma_u : \Gamma_r' : \Gamma_l'$ focused on 
$(A\,t)^\bot$ and $\vdash \Gamma_u : \Gamma_r'' : \Gamma_l''$ focused on $\nbang{r} \nquest{l} B\,t$. 

The first premise is then finished by an initial rule, which means that 
more constraints are generated. In particular, $A$ can be either in 
$\Gamma_u$, generating the constraints $\{\emp{\Gamma_r'},
\emp{\Gamma_l'}, \mctx{A}{\Gamma_u}\}$ or in one of the linear contexts,
generating the set of constraints $\{\emp{\Gamma_r'},
\elin{A}{\Gamma_l'}\}$ and $\{\elin{A}{\Gamma_r'}, \emp{\Gamma_l'}\}$. 
Moreover the closed leaf $\vdash \Gamma_u : \Gamma_r' : \Gamma_l'$ is 
added. 

For the second premise one introduces the $\nbang{r}$, which generates 
the constraint $\emp{\Gamma_l''}$ then the negative phase starts by 
adding $B\,t$ to the context $l$. Hence the macro-rule that introduces
the formula above has the following shape:
\[
 \infer={\vdash \Gamma_u : \Gamma_r, F : \Gamma_l}
{
\infer{\vdash \Gamma_u : \Gamma_r' : \Gamma_l'}{}
&
\vdash \Gamma_u : \Gamma_r'' : \Gamma_l'', B\,t
}
\]
and the constraints specify the relationship between the generic contexts 
and formulas. There are namely three sets of constraints, which means
that there are three possible macro-rules that can introduce the 
formula above:
\[
\begin{array}{c}
\left\{\begin{array}{c}
\emp{\Gamma_r'},\emp{\Gamma_l'}, \mctx{A}{\Gamma_u},
\union{\Gamma_r}{\Gamma_F}{\Gamma_r^1},\\
\elin{F}{\Gamma_F}, \union{\Gamma_r'}{\Gamma_r''}{\Gamma_r^1},
\union{\Gamma_l'}{\Gamma_l''}{\Gamma_l}, \emp{\Gamma_l''}
\end{array}\right\} \\\\

\left\{\begin{array}{c}
\emp{\Gamma_r'}, \elin{A}{\Gamma_l'},
\union{\Gamma_r}{\Gamma_F}{\Gamma_r^1},\\
\elin{F}{\Gamma_F}, \union{\Gamma_r'}{\Gamma_r''}{\Gamma_r^1},
\union{\Gamma_l'}{\Gamma_l''}{\Gamma_l}, \emp{\Gamma_l''}
\end{array}\right\} \\\\

\left\{\begin{array}{c}
\elin{A}{\Gamma_r'}, \emp{\Gamma_l'},
\union{\Gamma_r}{\Gamma_F}{\Gamma_r^1},\\
\elin{F}{\Gamma_F}, \union{\Gamma_r'}{\Gamma_r''}{\Gamma_r^1},
\union{\Gamma_l'}{\Gamma_l''}{\Gamma_l}, \emp{\Gamma_l''}
\end{array}\right\} 
\end{array}
\]
Notice that, for example, the auxiliary generic contexts $\Gamma_r^1$ and
$\Gamma_F$ do not appear in the shape of the macro-rule above, but it is
used for binding the contexts $\Gamma_r', \Gamma_r''$ and $\Gamma_r$
together. 



\subsection{Soundness and Completeness}

Given a satisfiable set of constraints $T$, it has a model $M$ which
can be described by a set of predicates $emp \backslash 1$,
$mctx \backslash 2$, $union \backslash 3$ and $elin \backslash 2$ that are true. 
Each valid instance of these predicates will trigger a rewritting rule as
described in Figure \ref{fig:rewriting}. Using
these rules it is possible to rewrite the generic
contexts of a derivation in terms of some formulas.

The generic contexts that are left in the end can be considered arbitrary sets
of formulas (and can be merged, or course, after all possible modifications).
%and the initial contexts
%(numbered 0). The generic contexts number 0 can all be considered arbitrary sets
%of formulas.

\begin{figure}
\begin{align*}
\emp{\Gamma} \qquad & \Gamma \rightarrow \cdot \\
\mctx{F}{\Gamma} \qquad & \Gamma \rightarrow \Gamma, F \\
\union{\Gamma'}{\Gamma''}{\Gamma} \qquad & \Gamma \rightarrow \Gamma', \Gamma'' \\
%\In{F}{\Gamma} \qquad & \Gamma \rightarrow \Gamma, F \\
\elin{F}{\Gamma} \qquad & \Gamma \rightarrow F \\
\top \qquad & \cdot, \Gamma \rightarrow \Gamma
\end{align*}
\caption{Rewritting rules used to transform the generic contexts into actual
sets of formulas.}
\label{fig:rewriting}
\end{figure}

We will prove that these rewriting rules converge in our case. For that, we need
the following lemma:

\begin{lemma}
\label{lemma:crit_pairs}
Let $M$ be any model of a set of constraints $T$. Then, the following always
holds:
\begin{align}
&\{ \emp{\Gamma}, \mctx{F}{\Gamma} \} \not\subseteq M \label{em} \\
&\{ \emp{\Gamma}, \elin{F}{\Gamma} \} \not\subseteq M \label{ee} \\
&\{ \emp{\Gamma}, \union{\Gamma'}{\Gamma''}{\Gamma} \} \subseteq M \Rightarrow \{ 
    \emp{\Gamma'}, \emp{\Gamma''} \} \subseteq M \label{eu} \\
&\{ \elin{F}{\Gamma}, \union{\Gamma'}{\Gamma''}{\Gamma} \} \subseteq M
\Rightarrow \{ \emp{\Gamma'}, \elin{F}{\Gamma''} \} \subseteq M \vee \nonumber \\
&\{ \emp{\Gamma''}, \elin{F}{\Gamma'} \} \subseteq M \label{elun}
\end{align}
\end{lemma}

\begin{proof}
To prove \ref{em} and \ref{ee}, note the clauses:
\[
\begin{array}{l}
  \In{F}{\Gamma} \subset \elin{F}{\Gamma}.\\
  \In{F}{\Gamma} \subset \mctx{F}{\Gamma}.\\
  \bot \subset \In{X}{\Gamma}, \form{X},\emp{\Gamma}.
\end{array}
\]
It is clear that if $\mctx{F}{\Gamma}$ or $\elin{F}{\Gamma}$ and $\emp{\Gamma}$
occur on the same set of constraints, it is unsatisfiable, thus, it has no
models.

Statement \ref{eu} is a direct consequence of the following clauses:
\[
\begin{array}{l}
\emp{\Gamma'} \subset \emp{\Gamma}, \union{\Gamma'}{\Gamma''}{\Gamma}. \\
\emp{\Gamma''} \subset \emp{\Gamma}, \union{\Gamma'}{\Gamma''}{\Gamma}.
\end{array}
\]

Statement \ref{elun} is a direct consequence of the following clauses:
\[
\begin{array}{l}
1 \{ \elin{F}{\Gamma'}, \elin{F}{\Gamma''} \} 1 \subset \elin{F}{\Gamma},
\union{\Gamma'}{\Gamma''}{\Gamma}. \\
1 \{ \emp{\Gamma'}, \emp{\Gamma''} \} 1 \subset \elin{F}{\Gamma},
\union{\Gamma'}{\Gamma''}{\Gamma}.
\end{array}
\]
\end{proof}

\begin{giselle}
Define $\mathcal{R}$:
For a model $M$ and a sequent $S$, $\mathcal{R}$ is defined as:
if M is empty, then, return S
else let el in M, then
  use rule corresponding to el to rewrite the contexts of S, obtaining S'
  return $\mathcal{R}(M \setminus \{el\}, S')$.
\end{giselle}

\begin{lemma}
\label{lemma:convergence}
Let $M$ be a model of a set of constraints $T$. The rewriting of contexts using
each element in $M$ once to apply the rules of Figure \ref{fig:rewriting} is convergent.
\end{lemma}

\begin{proof}
Since the model $M$ has a finite set of predicates, the rewriting always
terminates. Now, let's evaluate the critical pairs. By Lemma
\ref{lemma:crit_pairs}, we know that $\emp{\Gamma}$ never occurs on the same
model as $\elin{F}{\Gamma}$ or $\mctx{F}{\Gamma}$, and therefore we don't need to
treat them as critical pairs. 

If $\emp{\Gamma}$ and
$\union{\Gamma'}{\Gamma''}{\Gamma}$ occur on the same model, it is the case that
$\emp{\Gamma'}$ and $\emp{\Gamma''}$ also occur on this set, by Lemma
\ref{lemma:crit_pairs}. There are basically two possible options for rewriting:

$$\Gamma \xrightarrow{{\scriptstyle \emp{\Gamma}}} \cdot$$
$$\Gamma \xrightarrow{{\scriptstyle \union{\Gamma'}{\Gamma''}{\Gamma}}} \Gamma', \Gamma''
\xrightarrow{{\scriptstyle \emp{\Gamma'}}} \cdot, \Gamma''
\xrightarrow{{\scriptstyle \emp{\Gamma''}}} \cdot,
\cdot \rightarrow \cdot$$

The rule for $\emp{\Gamma}$ can be applied first, and we will get the empty set at once or
the rule for $\union{\Gamma'}{\Gamma''}{\Gamma}$ is applied and the $\Gamma'$
and $\Gamma''$ are rewritten as empty sets. In both cases, the result is the
same.

$\mctx{F}{\Gamma}$ and $\union{\Gamma'}{\Gamma''}{\Gamma}$ will not occur on the
same model since $\Gamma$ is unbounded on the first predicate but bounded on the
second.

If $\union{\Gamma'}{\Gamma''}{\Gamma}$ and $\elin{F}{\Gamma}$ occur on the same
model, by Lemma \ref{lemma:crit_pairs}, it is the case that one of the sets $\{
\emp{\Gamma'}, \elin{F}{\Gamma''} \}$ or $\{ \emp{\Gamma''}, \elin{F}{\Gamma'}
\}$ also occur in this model. There are basically three options for rewriting:

$$\Gamma \xrightarrow{{\scriptstyle \elin{F}{\Gamma}}} F$$
$$\Gamma \xrightarrow{{\scriptstyle \union{\Gamma'}{\Gamma''}{\Gamma}}} \Gamma', \Gamma''
\xrightarrow{{\scriptstyle \emp{\Gamma'}}} \cdot, \Gamma''
\xrightarrow{{\scriptstyle \elin{F}{\Gamma''}}} \cdot,
F \rightarrow F$$
$$\Gamma \xrightarrow{{\scriptstyle \union{\Gamma'}{\Gamma''}{\Gamma}}} \Gamma', \Gamma''
\xrightarrow{{\scriptstyle \emp{\Gamma''}}} \Gamma', \cdot
\xrightarrow{{\scriptstyle \elin{F}{\Gamma'}}} F,
\cdot \rightarrow F$$

One of the possible derivations is to rewrite
$\Gamma$ as $F$ at once, because of $\elin{F}{\Gamma}$. The other possible
derivation will rewrite $\Gamma$ as $\Gamma', \Gamma''$ and then one of these
symbols as the empty set and the other as $F$. In any of the possible
derivations, the result is the same.

We can conclude that this rewriting system converges under our assumptions.
\end{proof}

\begin{definition}[Correspondence]
Let $M$ be a model of a set of constraints $T$, $F$ be a formula, $\Oscr$ and
$\Cscr$ be a set of open and closed leaves and $S$ be a set of generic contexts.
Let $\Cscr'$, $\Oscr'$ and $S'$ be the multi-sets of closed,
open and end sequent contexts obtained from rewriting $\Cscr$, $\Oscr$ and
$S$ using $\mathcal{R}$ and $M$. 
Let $\sigma$ be a substitution mapping all
generic contexts in $\Cscr'$, $\Oscr'$ and $S'$ to multi-sets of formulas.
We say that $\langle \Oscr, \Cscr, S, F \rangle_M^{\sigma}$ corresponds to a derivation
$\Xi$ if its end sequent is $S'\sigma$ and the multi-sets of open and closed
leaves are exactly $\Oscr'\sigma$ and $\Cscr'\sigma$ respectively.
\end{definition}

\begin{theorem}
\label{thm:soundness}
(\textit{Soundness})
Let $F$ be a \sellf\ formula and $\mr{F}{S}{\Tscr} = \langle \Cscr, \Oscr,
\Tscr' \rangle$
be the macro-rule computed by the procedure described in Section 
\ref{sec:macro_def}. Let $T$ be a set of constraints in $\Tscr'$ and $M$ be an
arbitrary model of $T$. 
%Let $\Cscr'$, $\Oscr'$ and $S'$ be the multi-sets of closed,
%open and end sequent contexts obtained from rewriting $\Cscr$, $\Oscr$ and
%$S$ using $\mathcal{R}$ and $M$. 
Then, for any substitution $\sigma$ there
exists a derivation in \sellf\ corresponding to $\langle \Oscr, \Cscr, S, F
\rangle_M^{\sigma}$.  

%mapping all
%generic contexts in $\Cscr'$, $\Oscr'$ and $S'$ to multi-sets of formulas, there
%is a derivation in SELL whose end sequent is $S'\sigma$ and whose multi-sets of open and
%closed leaves are exactly $\Oscr'\sigma$ and $\Cscr'\sigma$ respectively.


%. If $M$ is a
%model of $T$, then the derivation using generic contexts can be transformed to
%an actual derivation in \sellf, in which the open and closed sequents
%correspond, respectively, to sequents in $\Oscr$ and $\Cscr$.

%Then, every satisfiable set $T \in \Tscr$ has a model $M$
%which can be translated to a derivation in \sellf\ when focusing in $F$.
\end{theorem}

\begin{proof}
Let $T$ be a satisfiable set of constraints in $\Tscr$ of a macro-rule for the 
formula $F$. Since this set is satisfiable, it has a model $M$ which can be
represented by a set of constraints that are true. For each of these valid
constraints, we rewrite the generic contexts of the derivation obtained as
described in Section \ref{sec:macro_def} using the rules of
Figure \ref{fig:rewriting}. Considering that the generic contexts left after
the rewritting is complete are arbitrary sets of formulas, we obtain a valid 
derivation in \sellf.

\begin{giselle}
It might be the
case that the set has more than one model, but the argument here should be valid for
all of them.
\end{giselle}

We prove this by induction on the size of the formula. 

The base case is when $F$ is atomic. If $F$ is a positive atomic formula $A$, the
following sets of constraints are generated:

\begin{enumerate}
  \item $T_u = \{\elin{A^{\bot}}{\Gamma}\} \cup \{ \emp{\Gamma_s} \mid \forall s
        \notin \Wscr\} $
  \item $T_l = \{\elin{A^{\bot}}{\Gamma_l} \} \cup \{ \emp{\Gamma_s} \mid \forall s
        \neq l \wedge s \notin \Wscr \} \cup \{ \emp{\Gamma} \}$
  \item $T_w = \{\mctx{A^{\bot}}{\Gamma_w} \} \cup \{ \emp{\Gamma_s} \mid \forall s
        \notin \Wscr \} \cup \{ \emp{\Gamma} \}$
\end{enumerate}

\begin{giselle}
TODO: explain the role of the models.
\end{giselle}

In all these cases, the indices of the generic contexts are not altered. All
that has to be done is rewrite some context $\Gamma_x$ as $A^{\bot}, \Gamma_x$
(for case 3) or $A^{\bot}$ (for case 1 and 2) and transform all other contexts
that cannot suffer weakening into empty sets. In that way, the derivation
becomes a valid application of \sellf's inital rule.

If $F$ is a negative atomic formula $A$, the rule $\alg{R\Uparrow}$ is applied and the
constraints 

$$T = \{ \elin{A}{\Gamma^{i+1}}, \union{\Gamma^i}{\Gamma^{i+1}}{\Gamma^{i+2}} \}$$

are generated. In this case, the derivation looks like this:

$$
\infer[R\Uparrow]{\genK : \Gamma^{i} \Uparrow A}{\genK :
\Gamma^{i+2} \Uparrow \cdot}
$$

Applying the rewritting rules defined previously, we obtain:

$$
\infer[R\Uparrow]{\genK : \Gamma^{i} \Uparrow A}{\genK :
\Gamma^{i}, A, \Gamma^{i+1} \Uparrow \cdot}
$$

Considering the leftover generic contexts as arbitrary sets of formulas:

$$
\infer[R\Uparrow]{\mathcal{K} : \Gamma \Uparrow A}{\mathcal{K} :
\Gamma, A \Uparrow \cdot}
$$

Which is a valid derivation in \sellf.
\end{proof}

\begin{comment}
1) Metodo (K): 
   model ---> tree
2) Metodo
   root sequent e formula --> set of set of constraints
             
{ K(M) | forall C in T. forall M. T |= M} = { set of derivations in SELLF introducing Root sequent}
\end{comment}

\begin{giselle}
Define T as a set of $in$ constraints for each formula on the contexts.

Define $\sigma$ as a mapping of any extra generic context to the empty set. 
\end{giselle}

\begin{theorem}
\label{thm:completeness}
(\textit{Completeness})
Let $F$ be a \sellf\ formula and $\Xi$ a derivation of sequent focused in $F$. 
Let $S$ be a generic context. Then there exists a set of constraints $\Tscr$ such that
$\mr{F}{S}{\Tscr} = \langle \Cscr, \Oscr, \Tscr' \rangle$ and there exists a
model $M$ for some $T \in \Tscr'$ and substitution $\sigma$, such that the $\langle \Oscr,
\Cscr, S, F \rangle_M^{\sigma}$ corresponds to $\Xi$. 
%obtained during the
%process described in Section \ref{sec:macro_def} that represents this
%derivation.
\end{theorem}

% inducao no tamanho da prova

% pegar todas as derivacoes em sellf -> 

Note that it is not required that the macro-rule is computed completely (until
no more operations are applicable). This is because all intermediate macro-rules
obtained during the process corresponds to a parcial derivation in \sellf. It is
important to notice that sets of constraints $T \in \Tscr$ that are
unsatisfiable correspond to derivations that fail. In fact, the inference rule
for a connective $c$ cannot be applied during proof search if the macro-rule it
generates makes unsatisfiable all previously satisfiable sets of constraints in
$\Tscr$.


%\paragraph{Soundness and Completeness}

%\textbf{VN: Perhaps to make things a bit more clearer, we need a notion of 
%$\alpha$-equivalence among sequent contexts.}

%Let $F$ be a \sellf\ formula. Then, every satisfiable instance of the macro-rules generated
%using the algorithm proposed corresponds to a derivation in \sellf\ when
%focusing on $F$.

%\vspace{0.5cm}

% G: Can I say that macro-rules with unsatisfiable constraints corresponds to
% derivations that fail at some point?

% Yes, this is what the soundness and completeness results will imply.

%  VN: Since there is no negation, there is a least model using standard 
%  Horn theory semantics. There are constraints, but those do not
% introduce 
% % models.
% 

% The following lemma is indeed not correct. The assumption that the 
% our constraints are Horn theories is false, due to the specification of 
% union constraints.

% \begin{lemma}
%  Let $\Tscr$ be a set of constraints of the form above. If $\Tscr$ is 
% satisfiable, then it has a model.
% \end{lemma}

% G: It's not a least model anymore. Check this proof.

%To formalize the soundness of our translation, that takes a
%model of a set of constraints, $\Cscr$, the root side-formulas, and the
%list of open leaves and returns the macro-rule using the least model of
%$\Cscr$. 

% TODO: verify this after making the necessary changes on the example.
\begin{comment}
We illustrate using our
previous example how this translation works. Consider the first set of
constraints. 

\[
\Tscr = \left\{\begin{array}{c}
\emp{\Gamma_r'},\emp{\Gamma_l'}, \mctx{A}{\Gamma_u},
\union{\Gamma_r}{\Gamma_F}{\Gamma_r^1},\\
\elin{F}{\Gamma_F}, \union{\Gamma_r'}{\Gamma_r''}{\Gamma_r^1},
\union{\Gamma_l'}{\Gamma_l''}{\Gamma_l}, \emp{\Gamma_l''}
\end{array}\right\} 
\]
These constraints are satisfiable. To determine its least model, we use the
smodels logic program engine~\cite{niemela97lpmnr}, obtaining the following
model:
\[
 \left\{\begin{array}{c}
\In{F}{\Gamma_r^1},\In{F}{\Gamma_F}, \In{F}{\Gamma_r''},
\In{A}{\Gamma_u}\\
\end{array}\right\}  \cup \Tscr
\]
We then use this model to trigger the rewrite rules of Figure
\ref{fig:rewritting} to the 
contexts appearing in the open leaves and the root side-formulas:

\[
 \begin{array}{l@{\quad}l}
  \emp{\Gamma} & \Gamma \rightarrow \emptyset\\
  \mctx{F}{\Gamma} & \Gamma \rightarrow \Gamma, F\\
  \union{\Gamma'}{\Gamma''}{\Gamma} & \Gamma \rightarrow \Gamma',
\Gamma''\\
  \In{F}{\Gamma} & \Gamma \rightarrow \Gamma, F\\
  \top & \emptyset, \Gamma \rightarrow \Gamma
 \end{array}
\]

The last rewrite is always applicable. Notice that for the other
constraints, such as \eqctx{\cdot}{\cdot}, we do not need rewrite
rules. These are used only for specifying the model.

\vspace{0.5cm}

\begin{lemma}
Let $mr(F) = \langle \Cscr, \Oscr, \Tscr \rangle$ be a macro-rule computed as shown
above. Let $M$ be an arbitrary model of a set of constraints in $\Tscr$
that is satisfiable.
% GR: which translation? the rewritting?
Then the macro-rule obtained by applying the translation shown above using 
$\Oscr, \Sscr$ and $M$ is derivable in \sellf\ by focusing on the
formula $F$. 
\end{lemma}

\begin{proof}
The proof follows by induction on the size of $F$.
The model of $\Tscr'$ is then used as an oracle on the non-deterministic
choices, such as how to do the splitting of contexts.

The first base case is when $F$ is a positive atomic formula. Then the set
of constraints
obtained from the macro-rule algorithm above is of the following form:
\[
 \In{F}{\Gamma} \quad \textrm{and} \quad \emp{\Gamma'}
\]
where $\Gamma$ is a context in the root side-formulas. All of these
constraints are satisfiable. Moreover, it returns an empty set of open leaves.
Then, the translation from (the) of a constraint model to a macro-rule
replaces a generic context $\Gamma$ appearing in an \In{F}{\Gamma}
constraint by $\Gamma, F$ and a generic context $\Gamma'$ appearing 
in a \emp{\Gamma'} by $\emptyset$. The
resulting 
macro-rule corresponds exactly to the a macro-rule when an initial rule is
applied. The second base case, which is when 
$F$ is a negative atomic formula, is straightforward, as no constraints 
are generated.

For the inductive cases, the most interesting case is when $F = F_1
\tensor F_2$. Assume that $\mr{F_1}{\Cscr_1}{\Oscr_1}{\Tscr_1}{\Sscr_1}$
and $\mr{F_2}{\Cscr_2}{\Oscr_2}{\Tscr_2}{\Sscr_2}$ are given, which encode
the macro-rules computed by introducing the subformulas of $F$. Then 
from the definition of the algorithm, the set of constraints $T$ from which
the model is obtained contains necessarily a set of constraints
constraint $T_1 \in \Tscr_1$ and $T_2 \in \Tscr_2$. Moreover, $T$
contains a set of \tsl{union} constraints specifying the split of linear
formulas among the branches.
Hence the \In{\cdot}{\cdot} elements of the model $M$ of $T$ specifies
how the formulas in the linear contexts of $\Sscr$ are to be split among
the linear contexts of $\Sscr_1$ and $\Sscr_2$. We use the same
split of formulas in the \sellf\ derivation. Finally, since the names used 
for the (linear) generic contexts appearing in one premise are different
from the names used in the other premise, 
the model $M$ restricted to the names appearing in the resulting
root side-formulas, $\Sscr_1$ and $\Sscr_2$, will also be a model of 
$T_1$ and $T_2$. Hence by induction hypothesis these are also valid
derivations in \sellf. 

Notice that the unbounded generic contexts do not 
cause a problem, although, their names are shared among the premises, since
they are subject to no constraints, except \tsl{mctx}, and that unbounded
contexts allow weakening.
\end{proof}

Completeness follows in a similar fashion, but by induction on the height
of derivations. It follows very closely the proof of the soundness lemma.
Notice that in the base case our algorithm a new set of constraints where
$F$ is any one of the subexponential 
contexts.

\begin{lemma}
Let $\Sscr$ be a root sequent context and $F$ a formula. Let $\Xi$ be a
macro-rule obtained in \sellf\ whose end sequent is $\vdash \Sscr
\Downarrow F$. Let $\mr{F}{\Cscr}{\Oscr}{\Tscr}{\Sscr}$ be the macro-rule
constructed using the algorithm above. Then there is a model $M$ of a 
set of constraints in $\Tscr$ that when used to rewrite $\Oscr$ the
resulting open premises are equivalent to those in $\Xi$. 
\end{lemma}

\end{comment}

\subsection{Putting two derivations together}

%Given a common root sequent context $\Sscr$, we will be interested in 
%determining whether if there is a proof where one formula in the
%meta-logic is focused 

Let $\mr{F_1}{\Sscr}{\Tscr_0}$ be the macro-rule of a formula $F_1$. After
finishing all possible applications of the rules, we obtain a final tuple
$\langle \Cscr_f, \Oscr_f, \Tscr_f \rangle$. Each sequent in $\Oscr_f$ is described only
by a set of generic contexts (since they are of the form $\genK : \Gamma^i
\Uparrow \cdot$), which are related through the constraints determined in
$\Tscr_f$.

Given this generic context and constraints, we can use them as an initial state
for the construction of another macro-rule of, say, a formula $F_2$:
$\mr{F_2}{\Sscr_f}{\Tscr_f}$. By the soundness and completeness of the
macro-rules, this is equivalent to a derivation in \sellf\ that focus on $F_1$
and later on $F_2$ in one of its branches.

Using this combination of macro-rules, it's possible to reason about the
permutation of inference rules of the system specified. This is described in
details in Section \ref{sec:permutations}

\section{Permutation Lemmas}
\label{sec:permutations}

\begin{comment}
\begin{giselle}
I though it was better to separate the concepts of macro-rules and
permutability. First we introduce what are macro-rules and prove it's
completeness and after that we present the problem of permutation of rules and
show how macro-rules can help us solve this automatically.
\end{giselle}

\begin{vivek} 
Define the problem of permuting two rules: Given a side-formula
context, and two formulas. Forall possible ways of 
introducing...
\end{vivek}

\begin{giselle}
There is a good definition of permutations on sequent calculus here:
http://www.lix.polytechnique.fr/~nguenot/pub/sd09.pdf Should we follow this
notation and definitions? I think they differ from what was defined in your
thesis Vivek.

Are we defining this for specifications of sequent calculus systems only?
\end{giselle}
\end{comment}

The proof of a theorem (or formula) in sequent calculus involves the systematic
application of inference rules to sequents. One major problem during proof
search on these systems is the existence of redundant derivations. By redundant
we mean derivations that can be obtained from each other just by a simple
reordering on the application of inference rules, without affecting the
end-sequent or the leaves. In this document, we will deal only with derivations
that are obtained by applying two inference rules. 

\begin{definition}[Permutability]
Let $\mathcal{S}$ be an end sequent and $\pi$ a derivation of $\mathcal{S}$
which consists of the application of rules $\alpha$ and $\beta$, in this
order, resulting on the set of sequents $\{\mathcal{S}_1, ..., \mathcal{S}_n\}$ as
leaves:

\[
\infer[\alpha]{\mathcal{S}}
{
  \mathcal{S}_1 &
  ... &
  \infer[\beta]{\mathcal{S}'}{\mathcal{S}_i & ... & \mathcal{S}_{i+k}} &
  ... &
  \mathcal{S}_n
}
\]

We say that rule $\alpha$ \textit{permutes over} rule $\beta$, denoted by
$\alpha/\beta$, if there exists a derivation:

\[
\infer[\beta]{\mathcal{S}}
{
  \mathcal{S'}_1 &
  ... &
  \infer[\alpha]{\mathcal{S}'}{\mathcal{S'}_i & ... & \mathcal{S'}_{i+j}} &
  ... &
  \infer[\alpha]{\mathcal{S}''}{\mathcal{S'}_r & ... & \mathcal{S'}_{r+j}} &
  ... &
  \mathcal{S'}_m
}
\]

Such that the existence of a proof for each $\mathcal{S}_1, ..., \mathcal{S}_n$
implies the existence of proofs for each $\mathcal{S'}_1, ..., \mathcal{S'}_m$.

We say that two rules $\alpha$ and $\beta$ are permutable if $\alpha/\beta$ and
$\beta/\alpha$.

\end{definition}

\subsection{Checking permutability}

In this section we show how it's possible to verify automatically if two rules
permute by using the macro-rules.

\begin{giselle}
Let $\alpha(A_1, ..., A_n)$ and $\beta(B_1, ..., B_m)$ be formulas with the
connectives $\alpha$ and $\beta$. We proceed by building the macro rule for
$\alpha$, thus obtaining a set of open leaves $\Oscr_{\alpha}$. Choose one leaf
$S$ from this set and use its generic contexts and constraints to build a macro rule
for $\beta$. The new set $\Oscr$ of open leaves is $\Oscr_{\beta} \cup
\Oscr_{\alpha} \setminus \{S\}$.

Let $M_1$ be the model satisfying the final set of constraints of this first
part.

Now, build the macro rule for $\beta$ first and then $\alpha$ proceeding the
same way. Let $M_2$ be the model satisfying the set of constraints of this
operation.

If $M_2 \Rightarrow M_1$, then a proof of the sequents on the first derivation
implies in a proof of the sequents on the second derivation (because $M_2$ is
``bigger'' then $M_1$ - remember: $x > 6 \Rightarrow x > 5$).

Otherwise, try to choose another open leaf for $\alpha$ on the second
derivation. If one leaf does not work, try every subset.
\end{giselle}

\begin{comment}
\begin{giselle} 
The idea of showing the permutability of two rules
$\alpha$ and $\beta$ is to construct the macro rules for both derivations and
check if they are equivalent.

If this is the case, I am putting the paragraph below in a new section so that
it can be explained and proved.

Formalize and prove soudness and completeness of permutation algorithm.
\end{giselle}

\paragraph{Constructing more complex derivations}
Above we have shown how to represent a single macro-rule using
simple constraints. However, in order to show the permutability of rules, 
we will need to construct derivations with height of two macro-rules.
Assume that $\Sscr$ is the root context and that we want to construct a
such a derivation introducing $\Sscr$, where the formula $F_1$ is
introduced last, resulting on the macro rule
$\mr{F_1}{\Cscr}{\Oscr}{\Tscr}{\Sscr}$, and $F_2$ is introduced (possibly
many times) on some of the open leaves of $\Oscr$, say the open leaves 
$\Sscr_{i_1}, \ldots, \Sscr_{i_n}$, where $i_j$ is its position in list
$\Oscr$.
Then we construct the macro-rules as shown above
$\mr{F_2}{\Cscr_{i_1}}{\Oscr_{i_1}}{\Tscr_{i_1}}{\Sscr_{i_1}}$, $\ldots,$
$\mr{F_2}{\Cscr_{i_n}}{\Oscr_{i_n}}{\Tscr_{i_n}}{\Sscr_{i_n}}$.
Let $\Cscr'$ be the list of closed leaves obtained by appending the lists 
$\Cscr_j$ and $\Oscr_j$ by replacing the ${i_j}$ element with the elements
in the list $\Oscr_{i_j}$. Let $\Tscr'$ be the set of sets of constraints, 
where the constraints above are merges, as in the case for tensor above.
We denote such
derivation as $\der{F_1}{F_2}{[i_1,
\ldots, i_n]}{\Cscr'}{\Oscr'}{\Tscr'}{\Sscr}$. 

Notice that it is not hard to generalize the definition above to describe 
derivations of with greater heights $\tsl{mr}^i$. However, this would
complicate unnecessarily notation and therefore we limit ourselves to 
$\tsl{mr}^2$.
\end{comment}



\bibliographystyle{abbrv}
\bibliography{permutation}


\end{document}
