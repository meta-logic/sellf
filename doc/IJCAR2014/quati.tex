\documentclass{llncs}
\usepackage[utf8]{inputenc}
\usepackage{amsmath}
\usepackage{amsfonts}
\usepackage{proof}

\title{Quati: An Automated Tool for Proving Permutation Lemmas}
% \title{Quati: checking permutation lemmas in sequent calculus}
\author{Vivek Nigam\inst{1} \and Giselle Reis\inst{2} \and Leonardo Lima\inst{1}}

\institute{Universidade Federal da Para\'{i}ba, Brazil
\and Technische Universit\"{a}t Wien, Austria
}

%% NOTE: 7 pages tops

%% TODO: references

\begin{document}
\maketitle

\begin{abstract}
The proof of many foundational results in structural proof theory, such as the
admissibility of the cut rule and the completeness of the focusing discipline,
rely on permutation lemmas. It is often a tedious and error prone task to prove
such lemmas as they involve many cases. This paper describes the tool Quati
which can automatically prove
%is an automated tool capable of proving 
a wide range of inference rule
permutations for a wide number of sequent calculus proof systems. Given a proof system
specification in the form of a theory in linear logic with subexponentials,
Quati outputs in \LaTeX\ the permutation transformations for which it was able
to prove correcteness and also the 
possible derivations for which it was not able to do so. As illustrated in this
paper, Quati's output is very similar to proof derivation figures one would
normally find in a proof theory book. 
\end{abstract}

\section{Introduction}

Permutation lemmas play an important role in proof theory. Many foundational
results about sequent calculus proof systems rely on the fact that some rules
permute over others. The most well-known of those is probably the admissibility of
the cut rule, which was proved for the systems \textbf{LK} and \textbf{LJ} by
Gentzen in 1934 \cite{gentzen}, and until today the same method is used to prove
cut elimination of various systems. Other important results that rely on the
permutability of rules are the mid-sequent theorem (or Herbrand's theorem)
\cite{herbrand} and the completeness of proof search strategies such as uniform
provability \cite{} and focusing \cite{andreoli}. Nevertheless, it is hard to
find complete proofs of these results with all the permutation cases listed.

As an example, consider the case of permuting $\vee_l$ over $\rightarrow_l$ in
the intuitionistic calculus \textbf{LJ}. In order to show whether these two
rules permute, one needs to check \emph{every possible case} in which
$\rightarrow_l$ occurs above $\vee_l$ in a derivation. When using a
multiplicative calculus, there are four possibilities for such derivation, two
allow a permutation of the rules while the other two don't\footnote{In an
additive calculus there are three possibilities, only one of them allow a
permutation.}. Here's one of each:
%% -> l / v l
{\scriptsize
\[
\infer[\vee_l]{\Gamma, \Gamma', \Gamma'', A \rightarrow B , P \vee Q \vdash F}{
  \deduce{\Gamma, P \vdash F}{\varphi_1\vspace{0.2cm}}
  &
  \infer[\rightarrow_l]{\Gamma', \Gamma'', A \rightarrow B , Q \vdash F}{
    \deduce{\Gamma' \vdash A}{\varphi_2\vspace{0.2cm}}
    &
    \deduce{\Gamma'', Q, B \vdash F}{\varphi_3\vspace{0.2cm}}
  }
}
\quad\rightsquigarrow\quad
\infer[\rightarrow_l]{\Gamma, \Gamma', \Gamma'', P \vee Q, A \rightarrow B
\vdash F}{
  \deduce{\Gamma' \vdash A}{\varphi_2\vspace{0.2cm}}
  &
  \infer[\vee_l]{\Gamma, \Gamma'', P \vee Q, B \vdash F}{
    \deduce{\Gamma, P \vdash F}{\varphi_1\vspace{0.2cm}}
    &
    \deduce{\Gamma'', B, Q \vdash F}{\varphi_3\vspace{0.2cm}}
  }
}
\]
}
%% -> l / v l
{\scriptsize
\[
\infer[\vee_l]{\Gamma, \Gamma', \Gamma'', A \rightarrow B , P \vee Q \vdash F}{
  \deduce{\Gamma, P \vdash F}{\varphi_1\vspace{0.2cm}}
  &
  \infer[\rightarrow_l]{\Gamma', \Gamma'', A \rightarrow B , Q \vdash F}{
    \deduce{\Gamma', Q \vdash A}{\varphi_2\vspace{0.2cm}}
    &
    \deduce{\Gamma'', B \vdash F}{\varphi_3\vspace{0.2cm}}
  }
}
\quad\rightsquigarrow\quad
?
\]
}
%
The combinatorial nature of permutation lemmas can be observed in this example.
While this illustrates only one case, proving the focusing discipline, for
instance, requires that the permutation of every pair of rules is checked. For a
system with $n$ rules, this amounts to $2^n$ cases.

The checking of all cases is a tedious task which can be error prone. And the
fact that the cases are rarely documented makes it hard for others to check the
correctness of the transformations. The cut-elimination result for
bi-intuitionistic logic, for example, given by Rauszer \cite{rauszer74studia}
was later found to be incorrect \cite{crolard01tcs} exactly because one of the
permutation lemmas was not true. An automated tool to check for these lemmas
would be therefore of great help.

In this paper we describe such a tool. Section \ref{sec:checking} explains
briefly the theoretical background that was implemented. Section \ref{sec:quati}
describes the actual implementation and explains, with an example, how to use
it. Section \ref{sec:conclusion} summarizes the results obtained and point the
directions of future work.

\section{Checking proof transformations}
\label{sec:checking}

% Explain briefly the method and cite other papers.
% This section should be small or even merged to the introduction.

Given a sequent calculus proof system, it was shown in \cite{ENTCS?} that it
is possible to encode it, under certain conditions, in linear logic with
subexponentials. This is a refinement of linear logic that allows and
arbitrary (finite) number of different types of the modalities $!$ and $?$. In
this encoding, each inference rule of the sequent calculus corresponds to a
formula in subexponential linear logic. This formula is called a \emph{bipole} because of its
particular nesting of operators, and its derivation, also called a bipole, is composed by exactly one
negative and one positive phase in the \emph{focusing} discipline. The details
of the encoding and the format of these derivations is out of the scope of this
paper and we refer the curious reader to \cite{llinda} and \cite{JLC paper} for
a deeper explanation. For the moment, it is enough to know that this encoding
has the highest level of adequacy \cite{adequacy??}, meaning that one inference
rule in the sequent calculus corresponds exactly to one bipole derivation in
focused linear logic with subexponentials (SELLF).

This correspondence allows the use of SELLF as a framework to reason
uniformly over a range of different proof systems for different logics. In
\cite{JLC?}, the authors show how to prove the admissibility of cut and atomic
initial rules\footnote{This method was implemented and released as the tool
TATU (\url{https://www.logic.at/staff/giselle/tatu/}).}, and in \cite{iclp
paper} it is shown how to use SELLF and answer-set programming to verify proof
transformations. Quati implements the ideas on the latter work to check
automatically whether two inference rules permute. 

\begin{definition}
Let $\alpha$ and $\beta$ be two inference rules in a sequent calculus proof
system. We say that $\alpha$ permutes over $\beta$ if \emph{every} possible
derivation that has an application of $\beta$ immediately followed by and
application of $\alpha$ (i.e. $\beta$ is above $\alpha$) can be transformed into
a derivation where $\alpha$ is followed by $\beta$.
\end{definition}

Using the SELLF specification of a rule, generic contexts and answer-set
programming, we can determine all the instances of an inference rule
application to a sequent \cite{iclp}. It is worth noting that this reasoning is
actually done on the linear logic level. So the derivation with generic contexts
is done in SELLF, and the models of the answer-set program specify bipole
derivations. But because of the specifications' high level of adequacy, we can
directly translate the bipole derivations to rules of the specified logic. A
rewriting system is presented in \cite{iclp} for this translation.

Given this algorithm, it is possible to specify all instances of $\alpha$ on a
generic sequent containing main formulas for $\alpha$ and $\beta$. For each of
these instances, we can also specify how $\beta$ can be applied on its premises.
Taking all possible combinations, we have the set $\mathcal{S}_1$ of
derivations in which $\beta$ is immediately above $\alpha$. Repeating the
procedure with the rules switched, we have the set $\mathcal{S}_2$ of
derivations in which $\alpha$ is immediately above $\beta$. To check the
permutation of $\alpha$ over $\beta$, one needs to check if a proof of each
derivation $d \in \mathcal{S}_1$ implies in a proof of some derivation $d' \in
\mathcal{S}_2$. The provability implication of derivations can also be
determined via an answer-set program described in \cite{iclp}.

\section{Quati}
\label{sec:quati}

% Implementation details and a demo session using the command line
% Link to the web

\section{Conclusion}
\label{sec:conclusion}

- mention the systems implemented and interesting cases of permutation

- cite the work with ramyaa about automating the specifications

- cite the automatic discovery of focusing systems


\end{document}
