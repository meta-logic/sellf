\documentclass{llncs}
\usepackage[utf8]{inputenc}
\usepackage{amsmath}
\usepackage{amsfonts}

\title{Quati: An Automated Tool for Proving Inference Rule Permutations}
% \title{Quati: checking permutation lemmas in sequent calculus}
\author{Vivek Nigam\inst{1} \and Giselle Reis\inst{2} \and Leonardo Lima\inst{1}}

\institute{Universidade Federal da Para\'{i}ba, Brazil
\and Technische Universit\"{a}t Wien, Austria
}

%% NOTE: 7 pages tops

\begin{document}
\maketitle

\begin{abstract}
The proof of many foundational results in Structural Proof Theory, such as the Admissibility of the Cut Rule and 
the Completeness of the Focusing Discipline, rely on Permutation Lemmas. It is often a tedious and error prone  
task to prove such lemmas as they involve many cases. This paper describes the tool Quati which is an automated tool capable of proving
a wide range of inference rule permutations for a wide number of proof systems. Given a proof system specification 
in the form of a theory in Linear Logic with Subexponentials, Quati
outputs in \LaTeX\ the permutation transformations for which it was able to prove correcteness and also the 
possible derivations for which it was not able to do so. As illustrated in this paper, Quati's output is very similar to 
proof derivation figures one would normally find in a Proof Theory book. 

\end{abstract}

\section{Introduction}

\section{Theoretical Background}

% Explain briefly

\subsection{Linear Logic and specification of systems}

\subsection{Answer set programming}

\section{Quati}

% Implementation details and a demo session (?)
% Link to the web

\section{Conclusion}

\end{document}
